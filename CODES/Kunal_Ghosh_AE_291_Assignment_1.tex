\documentclass[11pt]{article}

    \usepackage[breakable]{tcolorbox}
    \usepackage{parskip} % Stop auto-indenting (to mimic markdown behaviour)
    

    % Basic figure setup, for now with no caption control since it's done
    % automatically by Pandoc (which extracts ![](path) syntax from Markdown).
    \usepackage{graphicx}
    % Maintain compatibility with old templates. Remove in nbconvert 6.0
    \let\Oldincludegraphics\includegraphics
    % Ensure that by default, figures have no caption (until we provide a
    % proper Figure object with a Caption API and a way to capture that
    % in the conversion process - todo).
    \usepackage{caption}
    \DeclareCaptionFormat{nocaption}{}
    \captionsetup{format=nocaption,aboveskip=0pt,belowskip=0pt}

    \usepackage{float}
    \floatplacement{figure}{H} % forces figures to be placed at the correct location
    \usepackage{xcolor} % Allow colors to be defined
    \usepackage{enumerate} % Needed for markdown enumerations to work
    \usepackage{geometry} % Used to adjust the document margins
    \usepackage{amsmath} % Equations
    \usepackage{amssymb} % Equations
    \usepackage{textcomp} % defines textquotesingle
    % Hack from http://tex.stackexchange.com/a/47451/13684:
    \AtBeginDocument{%
        \def\PYZsq{\textquotesingle}% Upright quotes in Pygmentized code
    }
    \usepackage{upquote} % Upright quotes for verbatim code
    \usepackage{eurosym} % defines \euro

    \usepackage{iftex}
    \ifPDFTeX
        \usepackage[T1]{fontenc}
        \IfFileExists{alphabeta.sty}{
              \usepackage{alphabeta}
          }{
              \usepackage[mathletters]{ucs}
              \usepackage[utf8x]{inputenc}
          }
    \else
        \usepackage{fontspec}
        \usepackage{unicode-math}
    \fi

    \usepackage{fancyvrb} % verbatim replacement that allows latex
    \usepackage{grffile} % extends the file name processing of package graphics
                         % to support a larger range
    \makeatletter % fix for old versions of grffile with XeLaTeX
    \@ifpackagelater{grffile}{2019/11/01}
    {
      % Do nothing on new versions
    }
    {
      \def\Gread@@xetex#1{%
        \IfFileExists{"\Gin@base".bb}%
        {\Gread@eps{\Gin@base.bb}}%
        {\Gread@@xetex@aux#1}%
      }
    }
    \makeatother
    \usepackage[Export]{adjustbox} % Used to constrain images to a maximum size
    \adjustboxset{max size={0.9\linewidth}{0.9\paperheight}}

    % The hyperref package gives us a pdf with properly built
    % internal navigation ('pdf bookmarks' for the table of contents,
    % internal cross-reference links, web links for URLs, etc.)
    \usepackage{hyperref}
    % The default LaTeX title has an obnoxious amount of whitespace. By default,
    % titling removes some of it. It also provides customization options.
    \usepackage{titling}
    \usepackage{longtable} % longtable support required by pandoc >1.10
    \usepackage{booktabs}  % table support for pandoc > 1.12.2
    \usepackage{array}     % table support for pandoc >= 2.11.3
    \usepackage{calc}      % table minipage width calculation for pandoc >= 2.11.1
    \usepackage[inline]{enumitem} % IRkernel/repr support (it uses the enumerate* environment)
    \usepackage[normalem]{ulem} % ulem is needed to support strikethroughs (\sout)
                                % normalem makes italics be italics, not underlines
    \usepackage{mathrsfs}
    

    
    % Colors for the hyperref package
    \definecolor{urlcolor}{rgb}{0,.145,.698}
    \definecolor{linkcolor}{rgb}{.71,0.21,0.01}
    \definecolor{citecolor}{rgb}{.12,.54,.11}

    % ANSI colors
    \definecolor{ansi-black}{HTML}{3E424D}
    \definecolor{ansi-black-intense}{HTML}{282C36}
    \definecolor{ansi-red}{HTML}{E75C58}
    \definecolor{ansi-red-intense}{HTML}{B22B31}
    \definecolor{ansi-green}{HTML}{00A250}
    \definecolor{ansi-green-intense}{HTML}{007427}
    \definecolor{ansi-yellow}{HTML}{DDB62B}
    \definecolor{ansi-yellow-intense}{HTML}{B27D12}
    \definecolor{ansi-blue}{HTML}{208FFB}
    \definecolor{ansi-blue-intense}{HTML}{0065CA}
    \definecolor{ansi-magenta}{HTML}{D160C4}
    \definecolor{ansi-magenta-intense}{HTML}{A03196}
    \definecolor{ansi-cyan}{HTML}{60C6C8}
    \definecolor{ansi-cyan-intense}{HTML}{258F8F}
    \definecolor{ansi-white}{HTML}{C5C1B4}
    \definecolor{ansi-white-intense}{HTML}{A1A6B2}
    \definecolor{ansi-default-inverse-fg}{HTML}{FFFFFF}
    \definecolor{ansi-default-inverse-bg}{HTML}{000000}

    % common color for the border for error outputs.
    \definecolor{outerrorbackground}{HTML}{FFDFDF}

    % commands and environments needed by pandoc snippets
    % extracted from the output of `pandoc -s`
    \providecommand{\tightlist}{%
      \setlength{\itemsep}{0pt}\setlength{\parskip}{0pt}}
    \DefineVerbatimEnvironment{Highlighting}{Verbatim}{commandchars=\\\{\}}
    % Add ',fontsize=\small' for more characters per line
    \newenvironment{Shaded}{}{}
    \newcommand{\KeywordTok}[1]{\textcolor[rgb]{0.00,0.44,0.13}{\textbf{{#1}}}}
    \newcommand{\DataTypeTok}[1]{\textcolor[rgb]{0.56,0.13,0.00}{{#1}}}
    \newcommand{\DecValTok}[1]{\textcolor[rgb]{0.25,0.63,0.44}{{#1}}}
    \newcommand{\BaseNTok}[1]{\textcolor[rgb]{0.25,0.63,0.44}{{#1}}}
    \newcommand{\FloatTok}[1]{\textcolor[rgb]{0.25,0.63,0.44}{{#1}}}
    \newcommand{\CharTok}[1]{\textcolor[rgb]{0.25,0.44,0.63}{{#1}}}
    \newcommand{\StringTok}[1]{\textcolor[rgb]{0.25,0.44,0.63}{{#1}}}
    \newcommand{\CommentTok}[1]{\textcolor[rgb]{0.38,0.63,0.69}{\textit{{#1}}}}
    \newcommand{\OtherTok}[1]{\textcolor[rgb]{0.00,0.44,0.13}{{#1}}}
    \newcommand{\AlertTok}[1]{\textcolor[rgb]{1.00,0.00,0.00}{\textbf{{#1}}}}
    \newcommand{\FunctionTok}[1]{\textcolor[rgb]{0.02,0.16,0.49}{{#1}}}
    \newcommand{\RegionMarkerTok}[1]{{#1}}
    \newcommand{\ErrorTok}[1]{\textcolor[rgb]{1.00,0.00,0.00}{\textbf{{#1}}}}
    \newcommand{\NormalTok}[1]{{#1}}

    % Additional commands for more recent versions of Pandoc
    \newcommand{\ConstantTok}[1]{\textcolor[rgb]{0.53,0.00,0.00}{{#1}}}
    \newcommand{\SpecialCharTok}[1]{\textcolor[rgb]{0.25,0.44,0.63}{{#1}}}
    \newcommand{\VerbatimStringTok}[1]{\textcolor[rgb]{0.25,0.44,0.63}{{#1}}}
    \newcommand{\SpecialStringTok}[1]{\textcolor[rgb]{0.73,0.40,0.53}{{#1}}}
    \newcommand{\ImportTok}[1]{{#1}}
    \newcommand{\DocumentationTok}[1]{\textcolor[rgb]{0.73,0.13,0.13}{\textit{{#1}}}}
    \newcommand{\AnnotationTok}[1]{\textcolor[rgb]{0.38,0.63,0.69}{\textbf{\textit{{#1}}}}}
    \newcommand{\CommentVarTok}[1]{\textcolor[rgb]{0.38,0.63,0.69}{\textbf{\textit{{#1}}}}}
    \newcommand{\VariableTok}[1]{\textcolor[rgb]{0.10,0.09,0.49}{{#1}}}
    \newcommand{\ControlFlowTok}[1]{\textcolor[rgb]{0.00,0.44,0.13}{\textbf{{#1}}}}
    \newcommand{\OperatorTok}[1]{\textcolor[rgb]{0.40,0.40,0.40}{{#1}}}
    \newcommand{\BuiltInTok}[1]{{#1}}
    \newcommand{\ExtensionTok}[1]{{#1}}
    \newcommand{\PreprocessorTok}[1]{\textcolor[rgb]{0.74,0.48,0.00}{{#1}}}
    \newcommand{\AttributeTok}[1]{\textcolor[rgb]{0.49,0.56,0.16}{{#1}}}
    \newcommand{\InformationTok}[1]{\textcolor[rgb]{0.38,0.63,0.69}{\textbf{\textit{{#1}}}}}
    \newcommand{\WarningTok}[1]{\textcolor[rgb]{0.38,0.63,0.69}{\textbf{\textit{{#1}}}}}


    % Define a nice break command that doesn't care if a line doesn't already
    % exist.
    \def\br{\hspace*{\fill} \\* }
    % Math Jax compatibility definitions
    \def\gt{>}
    \def\lt{<}
    \let\Oldtex\TeX
    \let\Oldlatex\LaTeX
    \renewcommand{\TeX}{\textrm{\Oldtex}}
    \renewcommand{\LaTeX}{\textrm{\Oldlatex}}
    % Document parameters
    % Document title
    \title{Kunal\_Ghosh\_AE\_291\_Assignment\_1}
    
    
    
    
    
% Pygments definitions
\makeatletter
\def\PY@reset{\let\PY@it=\relax \let\PY@bf=\relax%
    \let\PY@ul=\relax \let\PY@tc=\relax%
    \let\PY@bc=\relax \let\PY@ff=\relax}
\def\PY@tok#1{\csname PY@tok@#1\endcsname}
\def\PY@toks#1+{\ifx\relax#1\empty\else%
    \PY@tok{#1}\expandafter\PY@toks\fi}
\def\PY@do#1{\PY@bc{\PY@tc{\PY@ul{%
    \PY@it{\PY@bf{\PY@ff{#1}}}}}}}
\def\PY#1#2{\PY@reset\PY@toks#1+\relax+\PY@do{#2}}

\@namedef{PY@tok@w}{\def\PY@tc##1{\textcolor[rgb]{0.73,0.73,0.73}{##1}}}
\@namedef{PY@tok@c}{\let\PY@it=\textit\def\PY@tc##1{\textcolor[rgb]{0.24,0.48,0.48}{##1}}}
\@namedef{PY@tok@cp}{\def\PY@tc##1{\textcolor[rgb]{0.61,0.40,0.00}{##1}}}
\@namedef{PY@tok@k}{\let\PY@bf=\textbf\def\PY@tc##1{\textcolor[rgb]{0.00,0.50,0.00}{##1}}}
\@namedef{PY@tok@kp}{\def\PY@tc##1{\textcolor[rgb]{0.00,0.50,0.00}{##1}}}
\@namedef{PY@tok@kt}{\def\PY@tc##1{\textcolor[rgb]{0.69,0.00,0.25}{##1}}}
\@namedef{PY@tok@o}{\def\PY@tc##1{\textcolor[rgb]{0.40,0.40,0.40}{##1}}}
\@namedef{PY@tok@ow}{\let\PY@bf=\textbf\def\PY@tc##1{\textcolor[rgb]{0.67,0.13,1.00}{##1}}}
\@namedef{PY@tok@nb}{\def\PY@tc##1{\textcolor[rgb]{0.00,0.50,0.00}{##1}}}
\@namedef{PY@tok@nf}{\def\PY@tc##1{\textcolor[rgb]{0.00,0.00,1.00}{##1}}}
\@namedef{PY@tok@nc}{\let\PY@bf=\textbf\def\PY@tc##1{\textcolor[rgb]{0.00,0.00,1.00}{##1}}}
\@namedef{PY@tok@nn}{\let\PY@bf=\textbf\def\PY@tc##1{\textcolor[rgb]{0.00,0.00,1.00}{##1}}}
\@namedef{PY@tok@ne}{\let\PY@bf=\textbf\def\PY@tc##1{\textcolor[rgb]{0.80,0.25,0.22}{##1}}}
\@namedef{PY@tok@nv}{\def\PY@tc##1{\textcolor[rgb]{0.10,0.09,0.49}{##1}}}
\@namedef{PY@tok@no}{\def\PY@tc##1{\textcolor[rgb]{0.53,0.00,0.00}{##1}}}
\@namedef{PY@tok@nl}{\def\PY@tc##1{\textcolor[rgb]{0.46,0.46,0.00}{##1}}}
\@namedef{PY@tok@ni}{\let\PY@bf=\textbf\def\PY@tc##1{\textcolor[rgb]{0.44,0.44,0.44}{##1}}}
\@namedef{PY@tok@na}{\def\PY@tc##1{\textcolor[rgb]{0.41,0.47,0.13}{##1}}}
\@namedef{PY@tok@nt}{\let\PY@bf=\textbf\def\PY@tc##1{\textcolor[rgb]{0.00,0.50,0.00}{##1}}}
\@namedef{PY@tok@nd}{\def\PY@tc##1{\textcolor[rgb]{0.67,0.13,1.00}{##1}}}
\@namedef{PY@tok@s}{\def\PY@tc##1{\textcolor[rgb]{0.73,0.13,0.13}{##1}}}
\@namedef{PY@tok@sd}{\let\PY@it=\textit\def\PY@tc##1{\textcolor[rgb]{0.73,0.13,0.13}{##1}}}
\@namedef{PY@tok@si}{\let\PY@bf=\textbf\def\PY@tc##1{\textcolor[rgb]{0.64,0.35,0.47}{##1}}}
\@namedef{PY@tok@se}{\let\PY@bf=\textbf\def\PY@tc##1{\textcolor[rgb]{0.67,0.36,0.12}{##1}}}
\@namedef{PY@tok@sr}{\def\PY@tc##1{\textcolor[rgb]{0.64,0.35,0.47}{##1}}}
\@namedef{PY@tok@ss}{\def\PY@tc##1{\textcolor[rgb]{0.10,0.09,0.49}{##1}}}
\@namedef{PY@tok@sx}{\def\PY@tc##1{\textcolor[rgb]{0.00,0.50,0.00}{##1}}}
\@namedef{PY@tok@m}{\def\PY@tc##1{\textcolor[rgb]{0.40,0.40,0.40}{##1}}}
\@namedef{PY@tok@gh}{\let\PY@bf=\textbf\def\PY@tc##1{\textcolor[rgb]{0.00,0.00,0.50}{##1}}}
\@namedef{PY@tok@gu}{\let\PY@bf=\textbf\def\PY@tc##1{\textcolor[rgb]{0.50,0.00,0.50}{##1}}}
\@namedef{PY@tok@gd}{\def\PY@tc##1{\textcolor[rgb]{0.63,0.00,0.00}{##1}}}
\@namedef{PY@tok@gi}{\def\PY@tc##1{\textcolor[rgb]{0.00,0.52,0.00}{##1}}}
\@namedef{PY@tok@gr}{\def\PY@tc##1{\textcolor[rgb]{0.89,0.00,0.00}{##1}}}
\@namedef{PY@tok@ge}{\let\PY@it=\textit}
\@namedef{PY@tok@gs}{\let\PY@bf=\textbf}
\@namedef{PY@tok@gp}{\let\PY@bf=\textbf\def\PY@tc##1{\textcolor[rgb]{0.00,0.00,0.50}{##1}}}
\@namedef{PY@tok@go}{\def\PY@tc##1{\textcolor[rgb]{0.44,0.44,0.44}{##1}}}
\@namedef{PY@tok@gt}{\def\PY@tc##1{\textcolor[rgb]{0.00,0.27,0.87}{##1}}}
\@namedef{PY@tok@err}{\def\PY@bc##1{{\setlength{\fboxsep}{\string -\fboxrule}\fcolorbox[rgb]{1.00,0.00,0.00}{1,1,1}{\strut ##1}}}}
\@namedef{PY@tok@kc}{\let\PY@bf=\textbf\def\PY@tc##1{\textcolor[rgb]{0.00,0.50,0.00}{##1}}}
\@namedef{PY@tok@kd}{\let\PY@bf=\textbf\def\PY@tc##1{\textcolor[rgb]{0.00,0.50,0.00}{##1}}}
\@namedef{PY@tok@kn}{\let\PY@bf=\textbf\def\PY@tc##1{\textcolor[rgb]{0.00,0.50,0.00}{##1}}}
\@namedef{PY@tok@kr}{\let\PY@bf=\textbf\def\PY@tc##1{\textcolor[rgb]{0.00,0.50,0.00}{##1}}}
\@namedef{PY@tok@bp}{\def\PY@tc##1{\textcolor[rgb]{0.00,0.50,0.00}{##1}}}
\@namedef{PY@tok@fm}{\def\PY@tc##1{\textcolor[rgb]{0.00,0.00,1.00}{##1}}}
\@namedef{PY@tok@vc}{\def\PY@tc##1{\textcolor[rgb]{0.10,0.09,0.49}{##1}}}
\@namedef{PY@tok@vg}{\def\PY@tc##1{\textcolor[rgb]{0.10,0.09,0.49}{##1}}}
\@namedef{PY@tok@vi}{\def\PY@tc##1{\textcolor[rgb]{0.10,0.09,0.49}{##1}}}
\@namedef{PY@tok@vm}{\def\PY@tc##1{\textcolor[rgb]{0.10,0.09,0.49}{##1}}}
\@namedef{PY@tok@sa}{\def\PY@tc##1{\textcolor[rgb]{0.73,0.13,0.13}{##1}}}
\@namedef{PY@tok@sb}{\def\PY@tc##1{\textcolor[rgb]{0.73,0.13,0.13}{##1}}}
\@namedef{PY@tok@sc}{\def\PY@tc##1{\textcolor[rgb]{0.73,0.13,0.13}{##1}}}
\@namedef{PY@tok@dl}{\def\PY@tc##1{\textcolor[rgb]{0.73,0.13,0.13}{##1}}}
\@namedef{PY@tok@s2}{\def\PY@tc##1{\textcolor[rgb]{0.73,0.13,0.13}{##1}}}
\@namedef{PY@tok@sh}{\def\PY@tc##1{\textcolor[rgb]{0.73,0.13,0.13}{##1}}}
\@namedef{PY@tok@s1}{\def\PY@tc##1{\textcolor[rgb]{0.73,0.13,0.13}{##1}}}
\@namedef{PY@tok@mb}{\def\PY@tc##1{\textcolor[rgb]{0.40,0.40,0.40}{##1}}}
\@namedef{PY@tok@mf}{\def\PY@tc##1{\textcolor[rgb]{0.40,0.40,0.40}{##1}}}
\@namedef{PY@tok@mh}{\def\PY@tc##1{\textcolor[rgb]{0.40,0.40,0.40}{##1}}}
\@namedef{PY@tok@mi}{\def\PY@tc##1{\textcolor[rgb]{0.40,0.40,0.40}{##1}}}
\@namedef{PY@tok@il}{\def\PY@tc##1{\textcolor[rgb]{0.40,0.40,0.40}{##1}}}
\@namedef{PY@tok@mo}{\def\PY@tc##1{\textcolor[rgb]{0.40,0.40,0.40}{##1}}}
\@namedef{PY@tok@ch}{\let\PY@it=\textit\def\PY@tc##1{\textcolor[rgb]{0.24,0.48,0.48}{##1}}}
\@namedef{PY@tok@cm}{\let\PY@it=\textit\def\PY@tc##1{\textcolor[rgb]{0.24,0.48,0.48}{##1}}}
\@namedef{PY@tok@cpf}{\let\PY@it=\textit\def\PY@tc##1{\textcolor[rgb]{0.24,0.48,0.48}{##1}}}
\@namedef{PY@tok@c1}{\let\PY@it=\textit\def\PY@tc##1{\textcolor[rgb]{0.24,0.48,0.48}{##1}}}
\@namedef{PY@tok@cs}{\let\PY@it=\textit\def\PY@tc##1{\textcolor[rgb]{0.24,0.48,0.48}{##1}}}

\def\PYZbs{\char`\\}
\def\PYZus{\char`\_}
\def\PYZob{\char`\{}
\def\PYZcb{\char`\}}
\def\PYZca{\char`\^}
\def\PYZam{\char`\&}
\def\PYZlt{\char`\<}
\def\PYZgt{\char`\>}
\def\PYZsh{\char`\#}
\def\PYZpc{\char`\%}
\def\PYZdl{\char`\$}
\def\PYZhy{\char`\-}
\def\PYZsq{\char`\'}
\def\PYZdq{\char`\"}
\def\PYZti{\char`\~}
% for compatibility with earlier versions
\def\PYZat{@}
\def\PYZlb{[}
\def\PYZrb{]}
\makeatother


    % For linebreaks inside Verbatim environment from package fancyvrb.
    \makeatletter
        \newbox\Wrappedcontinuationbox
        \newbox\Wrappedvisiblespacebox
        \newcommand*\Wrappedvisiblespace {\textcolor{red}{\textvisiblespace}}
        \newcommand*\Wrappedcontinuationsymbol {\textcolor{red}{\llap{\tiny$\m@th\hookrightarrow$}}}
        \newcommand*\Wrappedcontinuationindent {3ex }
        \newcommand*\Wrappedafterbreak {\kern\Wrappedcontinuationindent\copy\Wrappedcontinuationbox}
        % Take advantage of the already applied Pygments mark-up to insert
        % potential linebreaks for TeX processing.
        %        {, <, #, %, $, ' and ": go to next line.
        %        _, }, ^, &, >, - and ~: stay at end of broken line.
        % Use of \textquotesingle for straight quote.
        \newcommand*\Wrappedbreaksatspecials {%
            \def\PYGZus{\discretionary{\char`\_}{\Wrappedafterbreak}{\char`\_}}%
            \def\PYGZob{\discretionary{}{\Wrappedafterbreak\char`\{}{\char`\{}}%
            \def\PYGZcb{\discretionary{\char`\}}{\Wrappedafterbreak}{\char`\}}}%
            \def\PYGZca{\discretionary{\char`\^}{\Wrappedafterbreak}{\char`\^}}%
            \def\PYGZam{\discretionary{\char`\&}{\Wrappedafterbreak}{\char`\&}}%
            \def\PYGZlt{\discretionary{}{\Wrappedafterbreak\char`\<}{\char`\<}}%
            \def\PYGZgt{\discretionary{\char`\>}{\Wrappedafterbreak}{\char`\>}}%
            \def\PYGZsh{\discretionary{}{\Wrappedafterbreak\char`\#}{\char`\#}}%
            \def\PYGZpc{\discretionary{}{\Wrappedafterbreak\char`\%}{\char`\%}}%
            \def\PYGZdl{\discretionary{}{\Wrappedafterbreak\char`\$}{\char`\$}}%
            \def\PYGZhy{\discretionary{\char`\-}{\Wrappedafterbreak}{\char`\-}}%
            \def\PYGZsq{\discretionary{}{\Wrappedafterbreak\textquotesingle}{\textquotesingle}}%
            \def\PYGZdq{\discretionary{}{\Wrappedafterbreak\char`\"}{\char`\"}}%
            \def\PYGZti{\discretionary{\char`\~}{\Wrappedafterbreak}{\char`\~}}%
        }
        % Some characters . , ; ? ! / are not pygmentized.
        % This macro makes them "active" and they will insert potential linebreaks
        \newcommand*\Wrappedbreaksatpunct {%
            \lccode`\~`\.\lowercase{\def~}{\discretionary{\hbox{\char`\.}}{\Wrappedafterbreak}{\hbox{\char`\.}}}%
            \lccode`\~`\,\lowercase{\def~}{\discretionary{\hbox{\char`\,}}{\Wrappedafterbreak}{\hbox{\char`\,}}}%
            \lccode`\~`\;\lowercase{\def~}{\discretionary{\hbox{\char`\;}}{\Wrappedafterbreak}{\hbox{\char`\;}}}%
            \lccode`\~`\:\lowercase{\def~}{\discretionary{\hbox{\char`\:}}{\Wrappedafterbreak}{\hbox{\char`\:}}}%
            \lccode`\~`\?\lowercase{\def~}{\discretionary{\hbox{\char`\?}}{\Wrappedafterbreak}{\hbox{\char`\?}}}%
            \lccode`\~`\!\lowercase{\def~}{\discretionary{\hbox{\char`\!}}{\Wrappedafterbreak}{\hbox{\char`\!}}}%
            \lccode`\~`\/\lowercase{\def~}{\discretionary{\hbox{\char`\/}}{\Wrappedafterbreak}{\hbox{\char`\/}}}%
            \catcode`\.\active
            \catcode`\,\active
            \catcode`\;\active
            \catcode`\:\active
            \catcode`\?\active
            \catcode`\!\active
            \catcode`\/\active
            \lccode`\~`\~
        }
    \makeatother

    \let\OriginalVerbatim=\Verbatim
    \makeatletter
    \renewcommand{\Verbatim}[1][1]{%
        %\parskip\z@skip
        \sbox\Wrappedcontinuationbox {\Wrappedcontinuationsymbol}%
        \sbox\Wrappedvisiblespacebox {\FV@SetupFont\Wrappedvisiblespace}%
        \def\FancyVerbFormatLine ##1{\hsize\linewidth
            \vtop{\raggedright\hyphenpenalty\z@\exhyphenpenalty\z@
                \doublehyphendemerits\z@\finalhyphendemerits\z@
                \strut ##1\strut}%
        }%
        % If the linebreak is at a space, the latter will be displayed as visible
        % space at end of first line, and a continuation symbol starts next line.
        % Stretch/shrink are however usually zero for typewriter font.
        \def\FV@Space {%
            \nobreak\hskip\z@ plus\fontdimen3\font minus\fontdimen4\font
            \discretionary{\copy\Wrappedvisiblespacebox}{\Wrappedafterbreak}
            {\kern\fontdimen2\font}%
        }%

        % Allow breaks at special characters using \PYG... macros.
        \Wrappedbreaksatspecials
        % Breaks at punctuation characters . , ; ? ! and / need catcode=\active
        \OriginalVerbatim[#1,codes*=\Wrappedbreaksatpunct]%
    }
    \makeatother

    % Exact colors from NB
    \definecolor{incolor}{HTML}{303F9F}
    \definecolor{outcolor}{HTML}{D84315}
    \definecolor{cellborder}{HTML}{CFCFCF}
    \definecolor{cellbackground}{HTML}{F7F7F7}

    % prompt
    \makeatletter
    \newcommand{\boxspacing}{\kern\kvtcb@left@rule\kern\kvtcb@boxsep}
    \makeatother
    \newcommand{\prompt}[4]{
        {\ttfamily\llap{{\color{#2}[#3]:\hspace{3pt}#4}}\vspace{-\baselineskip}}
    }
    

    
    % Prevent overflowing lines due to hard-to-break entities
    \sloppy
    % Setup hyperref package
    \hypersetup{
      breaklinks=true,  % so long urls are correctly broken across lines
      colorlinks=true,
      urlcolor=urlcolor,
      linkcolor=linkcolor,
      citecolor=citecolor,
      }
    % Slightly bigger margins than the latex defaults
    
    \geometry{verbose,tmargin=1in,bmargin=1in,lmargin=1in,rmargin=1in}
    
    

\begin{document}
    
    \setcounter{secnumdepth}{0}
    
    

    
    \hypertarget{name-kunal-ghosh}{%
\section{Name: Kunal Ghosh}\label{name-kunal-ghosh}}

    \hypertarget{course-m.tech-aerospace-engineering}{%
\section{Course: M.Tech (Aerospace
Engineering)}\label{course-m.tech-aerospace-engineering}}

    \hypertarget{subject-ae-291-matrix-computations}{%
\section{Subject: AE 291 (Matrix
Computations)}\label{subject-ae-291-matrix-computations}}

    \hypertarget{sap-no.-6000007645}{%
\section{SAP No.: 6000007645}\label{sap-no.-6000007645}}

    \hypertarget{s.r.-no.-05-01-00-10-42-22-1-21061}{%
\section{S.R. No.:
05-01-00-10-42-22-1-21061}\label{s.r.-no.-05-01-00-10-42-22-1-21061}}

    \begin{center}\rule{0.5\linewidth}{0.5pt}\end{center}

    \hypertarget{importing-the-necessary-libraries}{%
\section{Importing the necessary
libraries}\label{importing-the-necessary-libraries}}

    \begin{tcolorbox}[breakable, size=fbox, boxrule=1pt, pad at break*=1mm,colback=cellbackground, colframe=cellborder]
\prompt{In}{incolor}{1}{\boxspacing}
\begin{Verbatim}[commandchars=\\\{\}]
\PY{k+kn}{import} \PY{n+nn}{numpy} \PY{k}{as} \PY{n+nn}{np}
\end{Verbatim}
\end{tcolorbox}

    \begin{tcolorbox}[breakable, size=fbox, boxrule=1pt, pad at break*=1mm,colback=cellbackground, colframe=cellborder]
\prompt{In}{incolor}{2}{\boxspacing}
\begin{Verbatim}[commandchars=\\\{\}]
\PY{k+kn}{import} \PY{n+nn}{pandas} \PY{k}{as} \PY{n+nn}{pd}
\end{Verbatim}
\end{tcolorbox}

    \begin{tcolorbox}[breakable, size=fbox, boxrule=1pt, pad at break*=1mm,colback=cellbackground, colframe=cellborder]
\prompt{In}{incolor}{3}{\boxspacing}
\begin{Verbatim}[commandchars=\\\{\}]
\PY{k+kn}{import} \PY{n+nn}{sys}
\end{Verbatim}
\end{tcolorbox}

    \begin{tcolorbox}[breakable, size=fbox, boxrule=1pt, pad at break*=1mm,colback=cellbackground, colframe=cellborder]
\prompt{In}{incolor}{4}{\boxspacing}
\begin{Verbatim}[commandchars=\\\{\}]
\PY{k+kn}{import} \PY{n+nn}{warnings}
\end{Verbatim}
\end{tcolorbox}

    \begin{tcolorbox}[breakable, size=fbox, boxrule=1pt, pad at break*=1mm,colback=cellbackground, colframe=cellborder]
\prompt{In}{incolor}{5}{\boxspacing}
\begin{Verbatim}[commandchars=\\\{\}]
\PY{n}{warnings}\PY{o}{.}\PY{n}{filterwarnings}\PY{p}{(}\PY{n}{action}\PY{o}{=}\PY{l+s+s1}{\PYZsq{}}\PY{l+s+s1}{ignore}\PY{l+s+s1}{\PYZsq{}}\PY{p}{,} \PY{n}{category}\PY{o}{=}\PY{n+ne}{UserWarning}\PY{p}{)}
\end{Verbatim}
\end{tcolorbox}

    \begin{center}\rule{0.5\linewidth}{0.5pt}\end{center}

    \hypertarget{problem-1}{%
\section{Problem (1):}\label{problem-1}}

    \hypertarget{write-a-function-that-implements-cholesky-decomposition-to-get-the-cholesky-factor-r-for-a-symmetric-positive-definite-matrix-a.-the-function-should-read-any-matrix-and-it-should-output-an-error-message-when-the-matrix-is-not-positive-definite.}{%
\section{Write a function that implements Cholesky decomposition to get
the Cholesky factor R for a (symmetric) positive definite matrix A. The
function should read any matrix, and it should output an error message
when the matrix is not positive
definite.}\label{write-a-function-that-implements-cholesky-decomposition-to-get-the-cholesky-factor-r-for-a-symmetric-positive-definite-matrix-a.-the-function-should-read-any-matrix-and-it-should-output-an-error-message-when-the-matrix-is-not-positive-definite.}}

    \begin{center}\rule{0.5\linewidth}{0.5pt}\end{center}

    \hypertarget{answer-1}{%
\section{Answer (1):}\label{answer-1}}

    \hypertarget{formula-to-calculate-the-cholesky-factor-r}{%
\section{Formula to calculate the Cholesky factor,
R}\label{formula-to-calculate-the-cholesky-factor-r}}

    Let, \[A = \begin{bmatrix}
    a_{11} & a_{12} & \cdots & a_{1n} \\
    a_{21} & a_{22} & \cdots & a_{2n} \\
    \vdots & \vdots & \ddots & \vdots \\
    a_{n1} & a_{n2} & \cdots & a_{nn}
\end{bmatrix}\]

    Also, \[R = \begin{bmatrix}
    r_{11} & r_{12} & r_{13} & \cdots & r_{1n} \\
    0 & r_{22} & r_{23} & \cdots & r_{2n} \\
    0 & 0 & r_{33} & \cdots & r_{3n} \\
    \vdots & \vdots & \vdots & \ddots & \vdots \\
    0 & 0 & 0 & \cdots & r_{nn}
\end{bmatrix}\]

    So, \[A = R^TR\]

    So, \[r_{11} = \sqrt{a_{11}}\]

    Also, \[r_{1j} = \frac{a_{1j}}{r_{11}}\] where, \(j \neq 1\) and
\(j = \{2,3,...,n\}\)

    So, if i \textgreater{} 1 and \(i = \{2,3,...,n\}\)
\[r_{ii} = \sqrt{a_{ii} - \left(\sum_{k = 1}^{i-1}r_{ki}^2\right)}\]

    So, if i \textgreater{} 1 and \(i = \{2,3,...,n\}\) and also,
\(j = i+1,i+2,...,n\)
\[r_{ij} = \frac{a_{ij} - \left(\sum_{k = 1}^{i-1}r_{ki}r_{kj}\right)}{r_{ii}}\]

    \hypertarget{pseudocode-to-calculate-the-cholesky-factor-r}{%
\section{Pseudocode to calculate the Cholesky factor,
R}\label{pseudocode-to-calculate-the-cholesky-factor-r}}

    for i = 1,2,\ldots{},n

    \(\:\:\:\:\:\:\)for k = 1,2,\ldots{},i-1 (NOT executed when i = 1)

    \(\:\:\:\:\:\:\)\(\:\:\:\:\:\:\)\(a_{ii} \leftarrow a_{ii} - a_{ki}^2\)

    \(\:\:\:\:\:\:\)\(a_{ii} \leq 0\),set error flag, exit (A is NOT
positive definite)

    \(\:\:\:\:\:\:\)\(a_{ii} \leftarrow \sqrt{a_{ii}}\) (This is \(r_{ii}\))

    \(\:\:\:\:\:\:\)for j = i+1,i+2,\ldots{},n (NOT executed when i = n)

    \(\:\:\:\:\:\:\)\(\:\:\:\:\:\:\)for k = 1,2,\ldots{},i-1 (NOT executed
when i = 1)

    \(\:\:\:\:\:\:\)\(\:\:\:\:\:\:\)\(\:\:\:\:\:\:\)\(a_{ij} \leftarrow a_{ij} - a_{ki}a_{kj}\)

    \(\:\:\:\:\:\:\)\(a_{ij} = \frac{a_{ij}}{a_{ii}}\) (This is \(r_{ij}\))

    \hypertarget{choleskya-function-would-take-symmetric-positive-definite-matix-a-as-input-and-return-cholesky-factor-r-as-output}{%
\section{Cholesky(A) function would take symmetric positive definite
matix (A) as input and return Cholesky factor (R) as
output}\label{choleskya-function-would-take-symmetric-positive-definite-matix-a-as-input-and-return-cholesky-factor-r-as-output}}

    \begin{tcolorbox}[breakable, size=fbox, boxrule=1pt, pad at break*=1mm,colback=cellbackground, colframe=cellborder]
\prompt{In}{incolor}{6}{\boxspacing}
\begin{Verbatim}[commandchars=\\\{\}]
\PY{k}{def} \PY{n+nf}{Cholesky}\PY{p}{(}\PY{n}{A}\PY{p}{)}\PY{p}{:}
    
    \PY{c+c1}{\PYZsh{} NOTE: Matrix A will be modified after the execution of this function }
    
    \PY{c+c1}{\PYZsh{} Error flag if the input matrix is NOT a square matrix}
    \PY{c+c1}{\PYZsh{} Checking if the number of rows and columns of the matrix A are equal or NOT}
    
    \PY{k}{if} \PY{n}{A}\PY{o}{.}\PY{n}{shape}\PY{p}{[}\PY{l+m+mi}{0}\PY{p}{]} \PY{o}{!=} \PY{n}{A}\PY{o}{.}\PY{n}{shape}\PY{p}{[}\PY{l+m+mi}{1}\PY{p}{]}\PY{p}{:}
        \PY{n+nb}{print}\PY{p}{(}\PY{l+s+sa}{f}\PY{l+s+s2}{\PYZdq{}}\PY{l+s+s2}{Cholesky(A): A matrix is NOT a square matrix}\PY{l+s+s2}{\PYZdq{}}\PY{p}{)}
        \PY{n}{sys}\PY{o}{.}\PY{n}{exit}\PY{p}{(}\PY{p}{)}
    \PY{k}{else}\PY{p}{:}
        \PY{c+c1}{\PYZsh{} Dimension of the matrix is stored in the variable n}
        \PY{n}{n} \PY{o}{=} \PY{n}{A}\PY{o}{.}\PY{n}{shape}\PY{p}{[}\PY{l+m+mi}{0}\PY{p}{]}
    
    \PY{c+c1}{\PYZsh{} Calculating the Cholesky factor (R)}
    \PY{k}{for} \PY{n}{i} \PY{o+ow}{in} \PY{n+nb}{range}\PY{p}{(}\PY{n}{n}\PY{p}{)}\PY{p}{:}
        \PY{k}{if} \PY{n}{i} \PY{o}{\PYZgt{}} \PY{l+m+mi}{0}\PY{p}{:}
            \PY{k}{for} \PY{n}{k} \PY{o+ow}{in} \PY{n+nb}{range}\PY{p}{(}\PY{n}{i}\PY{p}{)}\PY{p}{:}
                \PY{n}{A}\PY{p}{[}\PY{n}{i}\PY{p}{]}\PY{p}{[}\PY{n}{i}\PY{p}{]} \PY{o}{=} \PY{n}{A}\PY{p}{[}\PY{n}{i}\PY{p}{]}\PY{p}{[}\PY{n}{i}\PY{p}{]} \PY{o}{\PYZhy{}} \PY{p}{(}\PY{n}{A}\PY{p}{[}\PY{n}{k}\PY{p}{]}\PY{p}{[}\PY{n}{i}\PY{p}{]}\PY{o}{*}\PY{o}{*}\PY{l+m+mi}{2}\PY{p}{)}
        
        \PY{c+c1}{\PYZsh{} Error flag if the input matrix is NOT a positive definite matrix}
        
        \PY{k}{if} \PY{n}{A}\PY{p}{[}\PY{n}{i}\PY{p}{]}\PY{p}{[}\PY{n}{i}\PY{p}{]} \PY{o}{\PYZlt{}}\PY{o}{=} \PY{l+m+mi}{0}\PY{p}{:}
            \PY{n+nb}{print}\PY{p}{(}\PY{l+s+sa}{f}\PY{l+s+s2}{\PYZdq{}}\PY{l+s+s2}{Cholesky(A): A matrix is NOT a positive definite matrix}\PY{l+s+s2}{\PYZdq{}}\PY{p}{)}
            \PY{n}{sys}\PY{o}{.}\PY{n}{exit}\PY{p}{(}\PY{p}{)}
        \PY{k}{else}\PY{p}{:}
            \PY{n}{A}\PY{p}{[}\PY{n}{i}\PY{p}{]}\PY{p}{[}\PY{n}{i}\PY{p}{]} \PY{o}{=} \PY{n}{A}\PY{p}{[}\PY{n}{i}\PY{p}{]}\PY{p}{[}\PY{n}{i}\PY{p}{]}\PY{o}{*}\PY{o}{*}\PY{l+m+mf}{0.5}
        
        \PY{k}{for} \PY{n}{j} \PY{o+ow}{in} \PY{n+nb}{range}\PY{p}{(}\PY{n}{i}\PY{o}{+}\PY{l+m+mi}{1}\PY{p}{,}\PY{n}{n}\PY{p}{)}\PY{p}{:}
            \PY{k}{for} \PY{n}{k} \PY{o+ow}{in} \PY{n+nb}{range}\PY{p}{(}\PY{n}{i}\PY{p}{)}\PY{p}{:}
                \PY{n}{A}\PY{p}{[}\PY{n}{i}\PY{p}{]}\PY{p}{[}\PY{n}{j}\PY{p}{]} \PY{o}{=} \PY{n}{A}\PY{p}{[}\PY{n}{i}\PY{p}{]}\PY{p}{[}\PY{n}{j}\PY{p}{]} \PY{o}{\PYZhy{}} \PY{p}{(}\PY{n}{A}\PY{p}{[}\PY{n}{k}\PY{p}{]}\PY{p}{[}\PY{n}{i}\PY{p}{]}\PY{o}{*}\PY{n}{A}\PY{p}{[}\PY{n}{k}\PY{p}{]}\PY{p}{[}\PY{n}{j}\PY{p}{]}\PY{p}{)}
            \PY{n}{A}\PY{p}{[}\PY{n}{i}\PY{p}{]}\PY{p}{[}\PY{n}{j}\PY{p}{]} \PY{o}{=} \PY{n}{A}\PY{p}{[}\PY{n}{i}\PY{p}{]}\PY{p}{[}\PY{n}{j}\PY{p}{]}\PY{o}{/}\PY{n}{A}\PY{p}{[}\PY{n}{i}\PY{p}{]}\PY{p}{[}\PY{n}{i}\PY{p}{]}
    
    \PY{c+c1}{\PYZsh{} Declaring a new matrix to store the Cholesky factor(R)}
    \PY{n}{R} \PY{o}{=} \PY{n}{np}\PY{o}{.}\PY{n}{zeros}\PY{p}{(}\PY{n}{A}\PY{o}{.}\PY{n}{shape}\PY{p}{)}

    \PY{c+c1}{\PYZsh{} Storing the Cholesky factor to R}
    \PY{k}{for} \PY{n}{i} \PY{o+ow}{in} \PY{n+nb}{range}\PY{p}{(}\PY{n}{n}\PY{p}{)}\PY{p}{:}
        \PY{k}{for} \PY{n}{j} \PY{o+ow}{in} \PY{n+nb}{range}\PY{p}{(}\PY{n}{i}\PY{p}{,}\PY{n}{n}\PY{p}{)}\PY{p}{:}
            \PY{n}{R}\PY{p}{[}\PY{n}{i}\PY{p}{,}\PY{n}{j}\PY{p}{]} \PY{o}{=} \PY{n}{A}\PY{p}{[}\PY{n}{i}\PY{p}{,}\PY{n}{j}\PY{p}{]}
    
    \PY{c+c1}{\PYZsh{} Returning the Cholesky factor R}
    \PY{k}{return} \PY{n}{R}
\end{Verbatim}
\end{tcolorbox}

    \begin{center}\rule{0.5\linewidth}{0.5pt}\end{center}

    \hypertarget{problem-2}{%
\section{Problem (2):}\label{problem-2}}

    \hypertarget{write-functions-that-implement-forward-and-backward-substitutions-for-linear-systems-whose-coefficient-matrices-are-lower-and-upper-triangular-respectively.}{%
\section{Write functions that implement forward and backward
substitutions for linear systems whose coefficient matrices are lower
and upper triangular,
respectively.}\label{write-functions-that-implement-forward-and-backward-substitutions-for-linear-systems-whose-coefficient-matrices-are-lower-and-upper-triangular-respectively.}}

    \begin{center}\rule{0.5\linewidth}{0.5pt}\end{center}

    \hypertarget{answer-2}{%
\section{Answer (2):}\label{answer-2}}

    Let L be a lower triangular matrix. \[L = \begin{bmatrix}
    l_{11} & 0 & 0 & \cdots & 0 \\
    l_{21} & l_{22} & 0 & \cdots & 0 \\
    l_{31} & l_{32} & l_{33} & \cdots & 0 \\
    \vdots & \vdots & \vdots & \ddots & \vdots \\
    l_{n1} & l_{n2} & l_{n3} & \cdots & l_{nn}
\end{bmatrix}\]

    And b be a vector. \[b = \begin{bmatrix}
    b_{1}\\
    b_{2}\\
    b_{3}\\
    \vdots\\
    b_{n}
\end{bmatrix}\]

    So, Ly = b \[\begin{bmatrix}
    l_{11} & 0 & 0 & \cdots & 0 \\
    l_{21} & l_{22} & 0 & \cdots & 0 \\
    l_{31} & l_{32} & l_{33} & \cdots & 0 \\
    \vdots & \vdots & \vdots & \ddots & \vdots \\
    l_{n1} & l_{n2} & l_{n3} & \cdots & l_{nn}
\end{bmatrix} \begin{bmatrix}
    y_{1}\\
    y_{2}\\
    y_{3}\\
    \vdots\\
    y_{n}
\end{bmatrix} = \begin{bmatrix}
    b_{1}\\
    b_{2}\\
    b_{3}\\
    \vdots\\
    b_{n}
\end{bmatrix}\]

    \hypertarget{formula-for-forward-substitution}{%
\section{Formula for forward
substitution}\label{formula-for-forward-substitution}}

    \[y_1 = \frac{b_{1}}{l_{11}}\]

    For i \textgreater{} 1,
\[y_i = \frac{b_i - \left(\sum_{j = 1}^{i-1}l_{ij}y_{j}\right)}{l_{ii}}\]

    \hypertarget{pseudocode-for-forward-substitution}{%
\section{Pseudocode for forward
substitution}\label{pseudocode-for-forward-substitution}}

    for i = 1,2,\ldots{},n

    \(\qquad\)for j = 1,2,\ldots{},i-1 (NOT executed when i = 1)

    \(\qquad\)\(\qquad\)\(b_i\leftarrow b_i - (l_{ij}b_j)\)

    \(\qquad\)if \(l_{ii} = 0\), set error flag exit

    \(\qquad\)\(b_i\leftarrow \frac{b_i}{l_{ii}}\)

    \hypertarget{forward_lower_triangularlb-would-take-a-lower-triangular-matrix-l-and-the-corresponding-rhs-b-of-the-system-of-equations-as-inputs.-this-function-would-solve-them-using-forward-substitution-and-return-the-solution.-the-solution-will-be-returned-in-b.-so-the-vector-b-will-be-modified-after-the-execution-of-this-function.}{%
\section{Forward\_Lower\_Triangular(L,b) would take a lower triangular
matrix (L) and the corresponding RHS (b) of the system of equations as
inputs. This function would solve them using forward substitution and
return the solution. The solution will be returned in b. So, the vector
b will be modified after the execution of this
function.}\label{forward_lower_triangularlb-would-take-a-lower-triangular-matrix-l-and-the-corresponding-rhs-b-of-the-system-of-equations-as-inputs.-this-function-would-solve-them-using-forward-substitution-and-return-the-solution.-the-solution-will-be-returned-in-b.-so-the-vector-b-will-be-modified-after-the-execution-of-this-function.}}

    \begin{tcolorbox}[breakable, size=fbox, boxrule=1pt, pad at break*=1mm,colback=cellbackground, colframe=cellborder]
\prompt{In}{incolor}{7}{\boxspacing}
\begin{Verbatim}[commandchars=\\\{\}]
\PY{k}{def} \PY{n+nf}{Forward\PYZus{}Lower\PYZus{}Triangular}\PY{p}{(}\PY{n}{L}\PY{p}{,}\PY{n}{b}\PY{p}{)}\PY{p}{:}
    
    \PY{c+c1}{\PYZsh{} NOTE: Vector b will be modified after the execution of this function}
    
    \PY{c+c1}{\PYZsh{} Error flag if the input matrix is NOT a square matrix}
    \PY{c+c1}{\PYZsh{} Checking if the number of rows and columns of the matrix L are equal or NOT}
    
    \PY{k}{if} \PY{n}{L}\PY{o}{.}\PY{n}{shape}\PY{p}{[}\PY{l+m+mi}{0}\PY{p}{]} \PY{o}{!=} \PY{n}{L}\PY{o}{.}\PY{n}{shape}\PY{p}{[}\PY{l+m+mi}{1}\PY{p}{]}\PY{p}{:}
        \PY{n+nb}{print}\PY{p}{(}\PY{l+s+sa}{f}\PY{l+s+s2}{\PYZdq{}}\PY{l+s+s2}{Forward\PYZus{}Lower\PYZus{}Triangular(L,b): L matrix is NOT a square matrix}\PY{l+s+s2}{\PYZdq{}}\PY{p}{)}
        \PY{n}{sys}\PY{o}{.}\PY{n}{exit}\PY{p}{(}\PY{p}{)}
    \PY{k}{else}\PY{p}{:}
        \PY{c+c1}{\PYZsh{} Dimension of the matrix is stored in the variable n}
        \PY{n}{n} \PY{o}{=} \PY{n}{A}\PY{o}{.}\PY{n}{shape}\PY{p}{[}\PY{l+m+mi}{0}\PY{p}{]}
    
    \PY{c+c1}{\PYZsh{} Calculatig the solution}
    \PY{k}{for} \PY{n}{i} \PY{o+ow}{in} \PY{n+nb}{range}\PY{p}{(}\PY{n}{n}\PY{p}{)}\PY{p}{:}
        \PY{k}{if} \PY{n}{i} \PY{o}{\PYZgt{}} \PY{l+m+mi}{0}\PY{p}{:}
            \PY{k}{for} \PY{n}{j} \PY{o+ow}{in} \PY{n+nb}{range}\PY{p}{(}\PY{n}{i}\PY{p}{)}\PY{p}{:}
                \PY{n}{b}\PY{p}{[}\PY{n}{i}\PY{p}{]} \PY{o}{=} \PY{n}{b}\PY{p}{[}\PY{n}{i}\PY{p}{]} \PY{o}{\PYZhy{}} \PY{n}{L}\PY{p}{[}\PY{n}{i}\PY{p}{]}\PY{p}{[}\PY{n}{j}\PY{p}{]}\PY{o}{*}\PY{n}{b}\PY{p}{[}\PY{n}{j}\PY{p}{]}
        
        \PY{c+c1}{\PYZsh{} Error flag if the diagonal element of the input matrix, L is 0}
        \PY{k}{if} \PY{n}{L}\PY{p}{[}\PY{n}{i}\PY{p}{]}\PY{p}{[}\PY{n}{i}\PY{p}{]} \PY{o}{==} \PY{l+m+mi}{0}\PY{p}{:}
            \PY{n+nb}{print}\PY{p}{(}\PY{l+s+sa}{f}\PY{l+s+s2}{\PYZdq{}}\PY{l+s+s2}{Forward\PYZus{}Lower\PYZus{}Triangular(L,b): Diagonal element zero}\PY{l+s+s2}{\PYZdq{}}\PY{p}{)}
            \PY{n}{sys}\PY{o}{.}\PY{n}{exit}\PY{p}{(}\PY{p}{)}
        \PY{k}{else}\PY{p}{:}
            \PY{n}{b}\PY{p}{[}\PY{n}{i}\PY{p}{]} \PY{o}{=} \PY{n}{b}\PY{p}{[}\PY{n}{i}\PY{p}{]}\PY{o}{/}\PY{n}{L}\PY{p}{[}\PY{n}{i}\PY{p}{]}\PY{p}{[}\PY{n}{i}\PY{p}{]}
    
    \PY{c+c1}{\PYZsh{} Returning the solution}
    \PY{k}{return} \PY{n}{b}
\end{Verbatim}
\end{tcolorbox}

    Let U be an upper triangular matrix. \[U = \begin{bmatrix}
    u_{11} & u_{12} & u_{13} & \cdots & u_{1n} \\
    0 & u_{22} & u_{23} & \cdots & u_{2n} \\
    0 & 0 & u_{33} & \cdots & u_{3n} \\
    \vdots & \vdots & \vdots & \ddots & \vdots \\
    0 & 0 & 0 & \cdots & u_{nn}
\end{bmatrix}\]

    And b be a vector. \[b = \begin{bmatrix}
    b_{1}\\
    b_{2}\\
    b_{3}\\
    \vdots\\
    b_{n}
\end{bmatrix}\]

    So, Uz = b \[\begin{bmatrix}
    u_{11} & u_{12} & u_{13} & \cdots & u_{1n} \\
    0 & u_{22} & u_{23} & \cdots & u_{2n} \\
    0 & 0 & u_{33} & \cdots & u_{3n} \\
    \vdots & \vdots & \vdots & \ddots & \vdots \\
    0 & 0 & 0 & \cdots & u_{nn}
\end{bmatrix} \begin{bmatrix}
    z_{1}\\
    z_{2}\\
    z_{3}\\
    \vdots\\
    z_{n}
\end{bmatrix} = \begin{bmatrix}
    b_{1}\\
    b_{2}\\
    b_{3}\\
    \vdots\\
    b_{n}
\end{bmatrix}\]

    \hypertarget{formula-for-backward-substitution}{%
\section{Formula for backward
substitution}\label{formula-for-backward-substitution}}

    \[z_n = \frac{b_{n}}{u_{nn}}\]

    For i \textless{} n,
\[z_i = \frac{b_i - \left(\sum_{j = i+1}^{n}u_{ij}z_{j}\right)}{u_{ii}}\]

    \hypertarget{pseudocode-for-backward-substitution}{%
\section{Pseudocode for backward
substitution}\label{pseudocode-for-backward-substitution}}

    for i = n,n-1,\ldots{},1

    \(\qquad\)for j = i+1,i+2,\ldots{},n (NOT executed when i = n)

    \(\qquad\)\(\qquad\)\(b_i\leftarrow b_i - (u_{ij}b_j)\)

    \(\qquad\)if \(u_{ii} = 0\), set error flag exit

    \(\qquad\)\(b_i\leftarrow \frac{b_i}{u_{ii}}\)

    \hypertarget{backward_upper_triangularub-would-take-a-upper-triangular-matrix-u-and-the-corresponding-rhs-b-of-the-system-of-equations-as-inputs.-this-function-would-solve-them-using-backward-substitution-and-return-the-solution.-the-solution-will-be-returned-in-b.-so-the-vector-b-will-be-modified-after-the-execution-of-this-function.}{%
\section{Backward\_Upper\_Triangular(U,b) would take a upper triangular
matrix (U) and the corresponding RHS (b) of the system of equations as
inputs. This function would solve them using backward substitution and
return the solution. The solution will be returned in b. So, the vector
b will be modified after the execution of this
function.}\label{backward_upper_triangularub-would-take-a-upper-triangular-matrix-u-and-the-corresponding-rhs-b-of-the-system-of-equations-as-inputs.-this-function-would-solve-them-using-backward-substitution-and-return-the-solution.-the-solution-will-be-returned-in-b.-so-the-vector-b-will-be-modified-after-the-execution-of-this-function.}}

    \begin{tcolorbox}[breakable, size=fbox, boxrule=1pt, pad at break*=1mm,colback=cellbackground, colframe=cellborder]
\prompt{In}{incolor}{8}{\boxspacing}
\begin{Verbatim}[commandchars=\\\{\}]
\PY{k}{def} \PY{n+nf}{Backward\PYZus{}Upper\PYZus{}Triangular}\PY{p}{(}\PY{n}{U}\PY{p}{,}\PY{n}{b}\PY{p}{)}\PY{p}{:}
    
    \PY{c+c1}{\PYZsh{} NOTE: Vector b will be modified after the execution of this function}
    
    \PY{c+c1}{\PYZsh{} Error flag if the input matrix is NOT a square matrix}
    \PY{c+c1}{\PYZsh{} Checking if the number of rows and columns of the matrix U are equal or NOT}
    
    \PY{k}{if} \PY{n}{U}\PY{o}{.}\PY{n}{shape}\PY{p}{[}\PY{l+m+mi}{0}\PY{p}{]} \PY{o}{!=} \PY{n}{U}\PY{o}{.}\PY{n}{shape}\PY{p}{[}\PY{l+m+mi}{1}\PY{p}{]}\PY{p}{:}
        \PY{n+nb}{print}\PY{p}{(}\PY{l+s+sa}{f}\PY{l+s+s2}{\PYZdq{}}\PY{l+s+s2}{Backward\PYZus{}Upper\PYZus{}Triangular(U,b): U matrix is NOT a square matrix}\PY{l+s+s2}{\PYZdq{}}\PY{p}{)}
        \PY{n}{sys}\PY{o}{.}\PY{n}{exit}\PY{p}{(}\PY{p}{)}
    \PY{k}{else}\PY{p}{:}
        \PY{c+c1}{\PYZsh{} Dimension of the matrix is stored in the variable n}
        \PY{n}{n} \PY{o}{=} \PY{n}{A}\PY{o}{.}\PY{n}{shape}\PY{p}{[}\PY{l+m+mi}{0}\PY{p}{]}

    \PY{c+c1}{\PYZsh{} Calculatig the solution}
    \PY{k}{for} \PY{n}{i} \PY{o+ow}{in} \PY{n+nb}{range}\PY{p}{(}\PY{n}{n}\PY{o}{\PYZhy{}}\PY{l+m+mi}{1}\PY{p}{,}\PY{o}{\PYZhy{}}\PY{l+m+mi}{1}\PY{p}{,}\PY{o}{\PYZhy{}}\PY{l+m+mi}{1}\PY{p}{)}\PY{p}{:}
        \PY{k}{if} \PY{n}{i} \PY{o}{\PYZlt{}} \PY{p}{(}\PY{n}{n}\PY{o}{\PYZhy{}}\PY{l+m+mi}{1}\PY{p}{)}\PY{p}{:}
            \PY{k}{for} \PY{n}{j} \PY{o+ow}{in} \PY{n+nb}{range}\PY{p}{(}\PY{n}{i}\PY{o}{+}\PY{l+m+mi}{1}\PY{p}{,}\PY{n}{n}\PY{p}{)}\PY{p}{:}
                \PY{n}{b}\PY{p}{[}\PY{n}{i}\PY{p}{]} \PY{o}{=} \PY{n}{b}\PY{p}{[}\PY{n}{i}\PY{p}{]} \PY{o}{\PYZhy{}} \PY{n}{U}\PY{p}{[}\PY{n}{i}\PY{p}{]}\PY{p}{[}\PY{n}{j}\PY{p}{]}\PY{o}{*}\PY{n}{b}\PY{p}{[}\PY{n}{j}\PY{p}{]}

        \PY{c+c1}{\PYZsh{} Error flag if the diagonal element of the input matrix, U is 0}
        \PY{k}{if} \PY{n}{U}\PY{p}{[}\PY{n}{i}\PY{p}{]}\PY{p}{[}\PY{n}{i}\PY{p}{]} \PY{o}{==} \PY{l+m+mi}{0}\PY{p}{:}
            \PY{n+nb}{print}\PY{p}{(}\PY{l+s+sa}{f}\PY{l+s+s2}{\PYZdq{}}\PY{l+s+s2}{Backward\PYZus{}Upper\PYZus{}Triangular(U,b): Diagonal element zero}\PY{l+s+s2}{\PYZdq{}}\PY{p}{)}
            \PY{n}{sys}\PY{o}{.}\PY{n}{exit}\PY{p}{(}\PY{p}{)}
        \PY{k}{else}\PY{p}{:}
            \PY{n}{b}\PY{p}{[}\PY{n}{i}\PY{p}{]} \PY{o}{=} \PY{n}{b}\PY{p}{[}\PY{n}{i}\PY{p}{]}\PY{o}{/}\PY{n}{U}\PY{p}{[}\PY{n}{i}\PY{p}{]}\PY{p}{[}\PY{n}{i}\PY{p}{]}
    
    \PY{c+c1}{\PYZsh{} Returning the solution        }
    \PY{k}{return} \PY{n}{b}
\end{Verbatim}
\end{tcolorbox}

    \begin{center}\rule{0.5\linewidth}{0.5pt}\end{center}

    \hypertarget{problem-3}{%
\section{Problem (3):}\label{problem-3}}

    \hypertarget{write-a-function-that-solves-the-linear-system-axb-using-the-above-functions-taking-a-matrix-a-positive-definite-and-a-right-hand-side-vector-b.}{%
\section{Write a function that solves the linear system Ax=b using the
above functions, taking a matrix A (positive definite) and a right hand
side vector
b.}\label{write-a-function-that-solves-the-linear-system-axb-using-the-above-functions-taking-a-matrix-a-positive-definite-and-a-right-hand-side-vector-b.}}

    \begin{center}\rule{0.5\linewidth}{0.5pt}\end{center}

    \hypertarget{answer-3}{%
\section{Answer (3):}\label{answer-3}}

    \hypertarget{matrix_transposea-would-take-a-matrix-a-as-input-and-return-its-transpose-as-the-output}{%
\section{Matrix\_Transpose(A) would take a matrix A as input and return
its transpose as the
output}\label{matrix_transposea-would-take-a-matrix-a-as-input-and-return-its-transpose-as-the-output}}

\hypertarget{note-this-function-will-be-used-to-generate-positive-definite-matrix-from-any-random-matrix}{%
\section{(NOTE: This function will be used to generate positive definite
matrix from any random
matrix)}\label{note-this-function-will-be-used-to-generate-positive-definite-matrix-from-any-random-matrix}}

    \begin{tcolorbox}[breakable, size=fbox, boxrule=1pt, pad at break*=1mm,colback=cellbackground, colframe=cellborder]
\prompt{In}{incolor}{9}{\boxspacing}
\begin{Verbatim}[commandchars=\\\{\}]
\PY{k}{def} \PY{n+nf}{Matrix\PYZus{}Transpose}\PY{p}{(}\PY{n}{A}\PY{p}{)}\PY{p}{:}
    
    \PY{c+c1}{\PYZsh{} Declaring a new matrix (A\PYZus{}T) to store the transpose of A matrix}
    \PY{n}{A\PYZus{}T} \PY{o}{=} \PY{n}{np}\PY{o}{.}\PY{n}{zeros}\PY{p}{(}\PY{p}{(}\PY{n}{A}\PY{o}{.}\PY{n}{shape}\PY{p}{[}\PY{l+m+mi}{1}\PY{p}{]}\PY{p}{,}\PY{n}{A}\PY{o}{.}\PY{n}{shape}\PY{p}{[}\PY{l+m+mi}{0}\PY{p}{]}\PY{p}{)}\PY{p}{)}
    
    \PY{c+c1}{\PYZsh{} Calculating the transpose of matrix A}
    \PY{k}{for} \PY{n}{i} \PY{o+ow}{in} \PY{n+nb}{range}\PY{p}{(}\PY{n}{A}\PY{o}{.}\PY{n}{shape}\PY{p}{[}\PY{l+m+mi}{0}\PY{p}{]}\PY{p}{)}\PY{p}{:}
        \PY{k}{for} \PY{n}{j} \PY{o+ow}{in} \PY{n+nb}{range}\PY{p}{(}\PY{n}{A}\PY{o}{.}\PY{n}{shape}\PY{p}{[}\PY{l+m+mi}{1}\PY{p}{]}\PY{p}{)}\PY{p}{:}
            \PY{n}{A\PYZus{}T}\PY{p}{[}\PY{n}{i}\PY{p}{]}\PY{p}{[}\PY{n}{j}\PY{p}{]} \PY{o}{=} \PY{n}{A}\PY{p}{[}\PY{n}{j}\PY{p}{]}\PY{p}{[}\PY{n}{i}\PY{p}{]}
            
    \PY{c+c1}{\PYZsh{} Returning the transpose of matrix A}
    \PY{k}{return} \PY{n}{A\PYZus{}T}
\end{Verbatim}
\end{tcolorbox}

    As, A is a positive definite matrix, so, \[A = R^TR\] where, R =
Cholesky Factor of A

    Also, R is a upper triangular matrix.

    So, \(R^T\) is lower triangular matrix. Let, \[L = R^T\]

    So, \[A = LR\]

    As, we have \[Ax = b\]

    So, \[LRx = b\]

    Let, \[y = Rx\]

    So, \[Ly = b\]

    We can solve this for y using forward substitution, as L is a lower
triangular matrix.

    Once, we get the y, we don't need b after that. So, b will be modified.

    So, \[Rx = y\]

    We can solve this for x using backward substitution, as R is an upper
triangular matrix.

    So, we can obtain x for the linear system of equation \(Ax = b\),
provided A is a positive definte matrix.

    \hypertarget{linear_solverab-would-solve-a-system-of-equation-ax-b.-here-matrix-a-is-a-positive-definte-matrix.-also-the-vector-b-will-be-modified-after-the-execution-of-this-function}{%
\section{linear\_solver(A,b) would solve a system of equation Ax = b.
Here, matrix A is a positive definte matrix. Also, the vector b will be
modified after the execution of this
function}\label{linear_solverab-would-solve-a-system-of-equation-ax-b.-here-matrix-a-is-a-positive-definte-matrix.-also-the-vector-b-will-be-modified-after-the-execution-of-this-function}}

    \begin{tcolorbox}[breakable, size=fbox, boxrule=1pt, pad at break*=1mm,colback=cellbackground, colframe=cellborder]
\prompt{In}{incolor}{10}{\boxspacing}
\begin{Verbatim}[commandchars=\\\{\}]
\PY{k}{def} \PY{n+nf}{linear\PYZus{}solver}\PY{p}{(}\PY{n}{A}\PY{p}{,}\PY{n}{b}\PY{p}{)}\PY{p}{:}
    \PY{c+c1}{\PYZsh{} NOTE: Vector b will be modified after the execution of this function}
    
    \PY{c+c1}{\PYZsh{} Checking the compatibility of the dimensions of A and b}
    \PY{k}{if} \PY{n}{A}\PY{o}{.}\PY{n}{shape}\PY{p}{[}\PY{l+m+mi}{0}\PY{p}{]} \PY{o}{!=} \PY{n}{b}\PY{o}{.}\PY{n}{shape}\PY{p}{[}\PY{l+m+mi}{0}\PY{p}{]}\PY{p}{:}
        \PY{n+nb}{print}\PY{p}{(}\PY{l+s+sa}{f}\PY{l+s+s2}{\PYZdq{}}\PY{l+s+s2}{linear\PYZus{}solver(A,b): Dimensions of A and b are NOT compatible.}\PY{l+s+s2}{\PYZdq{}}\PY{p}{)}
        \PY{n}{sys}\PY{o}{.}\PY{n}{exit}\PY{p}{(}\PY{p}{)}
        
    \PY{c+c1}{\PYZsh{} Calculating the Cholesky factor of A}
    \PY{n}{R} \PY{o}{=} \PY{n}{Cholesky}\PY{p}{(}\PY{n}{A}\PY{p}{)}
    
    \PY{c+c1}{\PYZsh{} Calculating the transpose of the Cholesky factor of R}
    \PY{n}{L} \PY{o}{=} \PY{n}{Matrix\PYZus{}Transpose}\PY{p}{(}\PY{n}{R}\PY{p}{)}
    
    \PY{c+c1}{\PYZsh{} Solving the linear system of equation using forward substitution with lower triangular coefficient matrix, L}
    \PY{n}{b} \PY{o}{=} \PY{n}{Forward\PYZus{}Lower\PYZus{}Triangular}\PY{p}{(}\PY{n}{L}\PY{p}{,}\PY{n}{b}\PY{p}{)}
    
    \PY{c+c1}{\PYZsh{} Solving the linear system of equation using backward substitution with upper triangular coefficient matrix, U}
    \PY{n}{b} \PY{o}{=} \PY{n}{Backward\PYZus{}Upper\PYZus{}Triangular}\PY{p}{(}\PY{n}{R}\PY{p}{,}\PY{n}{b}\PY{p}{)}
    
    \PY{c+c1}{\PYZsh{} Returning the solution}
    \PY{k}{return} \PY{n}{b}
\end{Verbatim}
\end{tcolorbox}

    \begin{center}\rule{0.5\linewidth}{0.5pt}\end{center}

    \hypertarget{note-user-may-choose-to-directly-modify-a.csv-and-b.csv-to-input-the-coefficient-matrix-a-and-rhs-b-rather-than-generating-random-a-and-b.-then-the-user-should-skip-execution-until-snippet-below.}{%
\section{\texorpdfstring{NOTE: User may choose to directly modify
``A.csv'' and ``b.csv'' to input the coefficient matrix A and RHS b,
rather than generating random A and b. Then the user should skip
execution until \(^*\) snippet
below.}{NOTE: User may choose to directly modify ``A.csv'' and ``b.csv'' to input the coefficient matrix A and RHS b, rather than generating random A and b. Then the user should skip execution until \^{}* snippet below.}}\label{note-user-may-choose-to-directly-modify-a.csv-and-b.csv-to-input-the-coefficient-matrix-a-and-rhs-b-rather-than-generating-random-a-and-b.-then-the-user-should-skip-execution-until-snippet-below.}}

    \hypertarget{note-do-not-execute-the-snippets-immediately-below-dagger-if-a.csv-and-b.csv-is-given-by-the-user.}{%
\section{\texorpdfstring{NOTE: Do NOT execute the snippets immediately
below \(^\dagger\) if ``A.csv'' and ``b.csv'' is given by the
user.}{NOTE: Do NOT execute the snippets immediately below \^{}\textbackslash{}dagger if ``A.csv'' and ``b.csv'' is given by the user.}}\label{note-do-not-execute-the-snippets-immediately-below-dagger-if-a.csv-and-b.csv-is-given-by-the-user.}}

    \hypertarget{daggergenerating-a-random-matrix-m-using-numpy}{%
\section{\texorpdfstring{\(^\dagger\)Generating a random matrix, M using
numpy}{\^{}\textbackslash{}daggerGenerating a random matrix, M using numpy}}\label{daggergenerating-a-random-matrix-m-using-numpy}}

    \hypertarget{daggerdimension-of-matrix-m-can-be-modified-by-changing-m_n}{%
\section{\texorpdfstring{\(^\dagger\)Dimension of matrix M can be
modified by changing
M\_n}{\^{}\textbackslash{}daggerDimension of matrix M can be modified by changing M\_n}}\label{daggerdimension-of-matrix-m-can-be-modified-by-changing-m_n}}

    \[M\:\in \mathbb{R}^{M_n\times M_n}\]

    \begin{tcolorbox}[breakable, size=fbox, boxrule=1pt, pad at break*=1mm,colback=cellbackground, colframe=cellborder]
\prompt{In}{incolor}{11}{\boxspacing}
\begin{Verbatim}[commandchars=\\\{\}]
\PY{n}{M\PYZus{}n} \PY{o}{=} \PY{l+m+mi}{6}
\end{Verbatim}
\end{tcolorbox}

    \begin{tcolorbox}[breakable, size=fbox, boxrule=1pt, pad at break*=1mm,colback=cellbackground, colframe=cellborder]
\prompt{In}{incolor}{12}{\boxspacing}
\begin{Verbatim}[commandchars=\\\{\}]
\PY{n}{M} \PY{o}{=} \PY{n}{np}\PY{o}{.}\PY{n}{random}\PY{o}{.}\PY{n}{rand}\PY{p}{(}\PY{n}{M\PYZus{}n}\PY{p}{,}\PY{n}{M\PYZus{}n}\PY{p}{)}
\end{Verbatim}
\end{tcolorbox}

    \hypertarget{daggeras-m-matrix-must-be-non-singular}{%
\section{\texorpdfstring{\(^\dagger\)As M matrix must be
non-singular}{\^{}\textbackslash{}daggerAs M matrix must be non-singular}}\label{daggeras-m-matrix-must-be-non-singular}}

    \begin{tcolorbox}[breakable, size=fbox, boxrule=1pt, pad at break*=1mm,colback=cellbackground, colframe=cellborder]
\prompt{In}{incolor}{13}{\boxspacing}
\begin{Verbatim}[commandchars=\\\{\}]
\PY{k}{if} \PY{n}{np}\PY{o}{.}\PY{n}{linalg}\PY{o}{.}\PY{n}{det}\PY{p}{(}\PY{n}{M}\PY{p}{)} \PY{o}{==} \PY{l+m+mi}{0}\PY{p}{:}
    \PY{n+nb}{print}\PY{p}{(}\PY{l+s+s2}{\PYZdq{}}\PY{l+s+s2}{Please re\PYZhy{}generate the M matrix. A singular matrix, M was genenrated}\PY{l+s+s2}{\PYZdq{}}\PY{p}{)}
    \PY{n}{sys}\PY{o}{.}\PY{n}{exit}\PY{p}{(}\PY{p}{)}
\end{Verbatim}
\end{tcolorbox}

    \hypertarget{daggerdisplaying-the-matrix-m}{%
\section{\texorpdfstring{\(^\dagger\)Displaying the matrix
M}{\^{}\textbackslash{}daggerDisplaying the matrix M}}\label{daggerdisplaying-the-matrix-m}}

    \begin{tcolorbox}[breakable, size=fbox, boxrule=1pt, pad at break*=1mm,colback=cellbackground, colframe=cellborder]
\prompt{In}{incolor}{14}{\boxspacing}
\begin{Verbatim}[commandchars=\\\{\}]
\PY{n}{M}
\end{Verbatim}
\end{tcolorbox}

            \begin{tcolorbox}[breakable, size=fbox, boxrule=.5pt, pad at break*=1mm, opacityfill=0]
\prompt{Out}{outcolor}{14}{\boxspacing}
\begin{Verbatim}[commandchars=\\\{\}]
array([[0.24464033, 0.32890386, 0.67296616, 0.11117826, 0.01400661,
        0.4332117 ],
       [0.35383414, 0.65582155, 0.10782543, 0.9968977 , 0.66727074,
        0.34887289],
       [0.21904896, 0.82670995, 0.66667357, 0.72782795, 0.69605662,
        0.16114649],
       [0.34257744, 0.13813318, 0.47888485, 0.76742922, 0.39759538,
        0.22864822],
       [0.37246241, 0.09495505, 0.65927381, 0.8381055 , 0.01560623,
        0.83400147],
       [0.71334666, 0.55518648, 0.93109618, 0.27340465, 0.23606823,
        0.18951789]])
\end{Verbatim}
\end{tcolorbox}
        
    \hypertarget{daggergenerating-a-positive-definite-matix-a}{%
\section{\texorpdfstring{\(^\dagger\)Generating a positive definite
matix,
A}{\^{}\textbackslash{}daggerGenerating a positive definite matix, A}}\label{daggergenerating-a-positive-definite-matix-a}}

    \[A\:\in \mathbb{R}^{M_n\times M_n}\]

    \[A = M^TM\]

    \begin{tcolorbox}[breakable, size=fbox, boxrule=1pt, pad at break*=1mm,colback=cellbackground, colframe=cellborder]
\prompt{In}{incolor}{15}{\boxspacing}
\begin{Verbatim}[commandchars=\\\{\}]
\PY{n}{A} \PY{o}{=} \PY{p}{(}\PY{n}{M}\PY{o}{.}\PY{n}{T}\PY{p}{)}\PY{n+nd}{@M}
\end{Verbatim}
\end{tcolorbox}

    \hypertarget{daggerstoring-the-number-of-rows-of-matrix-a}{%
\section{\texorpdfstring{\(^\dagger\)Storing the number of rows of
matrix
A}{\^{}\textbackslash{}daggerStoring the number of rows of matrix A}}\label{daggerstoring-the-number-of-rows-of-matrix-a}}

    \begin{tcolorbox}[breakable, size=fbox, boxrule=1pt, pad at break*=1mm,colback=cellbackground, colframe=cellborder]
\prompt{In}{incolor}{16}{\boxspacing}
\begin{Verbatim}[commandchars=\\\{\}]
\PY{n}{n} \PY{o}{=} \PY{n}{A}\PY{o}{.}\PY{n}{shape}\PY{p}{[}\PY{l+m+mi}{0}\PY{p}{]}
\end{Verbatim}
\end{tcolorbox}

    \hypertarget{daggergenerating-a-random-rhs-b-for-the-system-of-equation-ax-b}{%
\section{\texorpdfstring{\(^\dagger\)Generating a random RHS (b) for the
system of equation Ax =
b}{\^{}\textbackslash{}daggerGenerating a random RHS (b) for the system of equation Ax = b}}\label{daggergenerating-a-random-rhs-b-for-the-system-of-equation-ax-b}}

    \[b\:\in \mathbb{R}^{M_n}\]

    \begin{tcolorbox}[breakable, size=fbox, boxrule=1pt, pad at break*=1mm,colback=cellbackground, colframe=cellborder]
\prompt{In}{incolor}{17}{\boxspacing}
\begin{Verbatim}[commandchars=\\\{\}]
\PY{n}{b} \PY{o}{=} \PY{n}{np}\PY{o}{.}\PY{n}{random}\PY{o}{.}\PY{n}{rand}\PY{p}{(}\PY{n}{n}\PY{p}{)}
\end{Verbatim}
\end{tcolorbox}

    \hypertarget{daggerconverting-a-to-a-pandas-dataframe}{%
\section{\texorpdfstring{\(^\dagger\)Converting A to a pandas
dataframe}{\^{}\textbackslash{}daggerConverting A to a pandas dataframe}}\label{daggerconverting-a-to-a-pandas-dataframe}}

    \begin{tcolorbox}[breakable, size=fbox, boxrule=1pt, pad at break*=1mm,colback=cellbackground, colframe=cellborder]
\prompt{In}{incolor}{18}{\boxspacing}
\begin{Verbatim}[commandchars=\\\{\}]
\PY{n}{A\PYZus{}file} \PY{o}{=} \PY{n}{pd}\PY{o}{.}\PY{n}{DataFrame}\PY{p}{(}\PY{n}{A}\PY{p}{)}
\end{Verbatim}
\end{tcolorbox}

    \hypertarget{daggerconverting-b-to-a-pandas-dataframe}{%
\section{\texorpdfstring{\(^\dagger\)Converting b to a pandas
dataframe}{\^{}\textbackslash{}daggerConverting b to a pandas dataframe}}\label{daggerconverting-b-to-a-pandas-dataframe}}

    \begin{tcolorbox}[breakable, size=fbox, boxrule=1pt, pad at break*=1mm,colback=cellbackground, colframe=cellborder]
\prompt{In}{incolor}{19}{\boxspacing}
\begin{Verbatim}[commandchars=\\\{\}]
\PY{n}{b\PYZus{}file} \PY{o}{=} \PY{n}{pd}\PY{o}{.}\PY{n}{DataFrame}\PY{p}{(}\PY{n}{b}\PY{p}{)}
\end{Verbatim}
\end{tcolorbox}

    \hypertarget{daggerwriting-a-to-a-csv-file}{%
\section{\texorpdfstring{\(^\dagger\)Writing A to a csv
file}{\^{}\textbackslash{}daggerWriting A to a csv file}}\label{daggerwriting-a-to-a-csv-file}}

    \begin{tcolorbox}[breakable, size=fbox, boxrule=1pt, pad at break*=1mm,colback=cellbackground, colframe=cellborder]
\prompt{In}{incolor}{20}{\boxspacing}
\begin{Verbatim}[commandchars=\\\{\}]
\PY{n}{A\PYZus{}file}\PY{o}{.}\PY{n}{to\PYZus{}csv}\PY{p}{(}\PY{l+s+s2}{\PYZdq{}}\PY{l+s+s2}{A.csv}\PY{l+s+s2}{\PYZdq{}}\PY{p}{,}\PY{n}{index}\PY{o}{=}\PY{k+kc}{None}\PY{p}{,}\PY{n}{header}\PY{o}{=}\PY{k+kc}{None}\PY{p}{)}
\end{Verbatim}
\end{tcolorbox}

    \hypertarget{daggerwriting-b-to-a-csv-file}{%
\section{\texorpdfstring{\(^\dagger\)Writing b to a csv
file}{\^{}\textbackslash{}daggerWriting b to a csv file}}\label{daggerwriting-b-to-a-csv-file}}

    \begin{tcolorbox}[breakable, size=fbox, boxrule=1pt, pad at break*=1mm,colback=cellbackground, colframe=cellborder]
\prompt{In}{incolor}{21}{\boxspacing}
\begin{Verbatim}[commandchars=\\\{\}]
\PY{n}{b\PYZus{}file}\PY{o}{.}\PY{n}{to\PYZus{}csv}\PY{p}{(}\PY{l+s+s2}{\PYZdq{}}\PY{l+s+s2}{b.csv}\PY{l+s+s2}{\PYZdq{}}\PY{p}{,}\PY{n}{index}\PY{o}{=}\PY{k+kc}{None}\PY{p}{,}\PY{n}{header}\PY{o}{=}\PY{k+kc}{None}\PY{p}{)}
\end{Verbatim}
\end{tcolorbox}

    \begin{center}\rule{0.5\linewidth}{0.5pt}\end{center}

    \hypertarget{reading-from-the-input-files}{%
\section{\texorpdfstring{Reading from the input
files\(^*\)}{Reading from the input files\^{}*}}\label{reading-from-the-input-files}}

    \hypertarget{reading-input-file-a.csv-for-the-coefficient-matrix-a}{%
\section{Reading input file ``A.csv'' for the coefficient matrix
A}\label{reading-input-file-a.csv-for-the-coefficient-matrix-a}}

\hypertarget{this-can-be-modified-by-the-user}{%
\section{(This can be modified by the
user)}\label{this-can-be-modified-by-the-user}}

    \begin{tcolorbox}[breakable, size=fbox, boxrule=1pt, pad at break*=1mm,colback=cellbackground, colframe=cellborder]
\prompt{In}{incolor}{22}{\boxspacing}
\begin{Verbatim}[commandchars=\\\{\}]
\PY{n}{A} \PY{o}{=} \PY{n}{pd}\PY{o}{.}\PY{n}{read\PYZus{}csv}\PY{p}{(}\PY{l+s+s2}{\PYZdq{}}\PY{l+s+s2}{A.csv}\PY{l+s+s2}{\PYZdq{}}\PY{p}{,}\PY{n}{header}\PY{o}{=}\PY{k+kc}{None}\PY{p}{)}
\end{Verbatim}
\end{tcolorbox}

    \hypertarget{converting-the-a-matrix-from-pandas-dataframe-to-numpy-array}{%
\section{Converting the A matrix from pandas dataframe to numpy
array}\label{converting-the-a-matrix-from-pandas-dataframe-to-numpy-array}}

    \begin{tcolorbox}[breakable, size=fbox, boxrule=1pt, pad at break*=1mm,colback=cellbackground, colframe=cellborder]
\prompt{In}{incolor}{23}{\boxspacing}
\begin{Verbatim}[commandchars=\\\{\}]
\PY{n}{A} \PY{o}{=} \PY{n}{A}\PY{o}{.}\PY{n}{to\PYZus{}numpy}\PY{p}{(}\PY{n}{dtype} \PY{o}{=} \PY{n}{np}\PY{o}{.}\PY{n}{float64}\PY{p}{)}
\end{Verbatim}
\end{tcolorbox}

    \hypertarget{reading-input-file-b.csv-for-the-rhs-b}{%
\section{Reading input file ``b.csv'' for the RHS
b}\label{reading-input-file-b.csv-for-the-rhs-b}}

\hypertarget{this-can-be-modified-by-the-user}{%
\section{(This can be modified by the
user)}\label{this-can-be-modified-by-the-user}}

    \begin{tcolorbox}[breakable, size=fbox, boxrule=1pt, pad at break*=1mm,colback=cellbackground, colframe=cellborder]
\prompt{In}{incolor}{24}{\boxspacing}
\begin{Verbatim}[commandchars=\\\{\}]
\PY{n}{b} \PY{o}{=} \PY{n}{pd}\PY{o}{.}\PY{n}{read\PYZus{}csv}\PY{p}{(}\PY{l+s+s2}{\PYZdq{}}\PY{l+s+s2}{b.csv}\PY{l+s+s2}{\PYZdq{}}\PY{p}{,}\PY{n}{header}\PY{o}{=}\PY{k+kc}{None}\PY{p}{)}
\end{Verbatim}
\end{tcolorbox}

    \hypertarget{converting-the-b-vector-from-pandas-dataframe-to-numpy-array}{%
\section{Converting the b vector from pandas dataframe to numpy
array}\label{converting-the-b-vector-from-pandas-dataframe-to-numpy-array}}

    \begin{tcolorbox}[breakable, size=fbox, boxrule=1pt, pad at break*=1mm,colback=cellbackground, colframe=cellborder]
\prompt{In}{incolor}{25}{\boxspacing}
\begin{Verbatim}[commandchars=\\\{\}]
\PY{n}{b} \PY{o}{=} \PY{n}{b}\PY{o}{.}\PY{n}{to\PYZus{}numpy}\PY{p}{(}\PY{n}{dtype} \PY{o}{=} \PY{n}{np}\PY{o}{.}\PY{n}{float64}\PY{p}{)}
\end{Verbatim}
\end{tcolorbox}

    \hypertarget{calculating-the-solution-of-ax-b-using-inbuilt-numpy-functions}{%
\section{Calculating the solution of Ax = b, using inbuilt numpy
functions}\label{calculating-the-solution-of-ax-b-using-inbuilt-numpy-functions}}

    \begin{tcolorbox}[breakable, size=fbox, boxrule=1pt, pad at break*=1mm,colback=cellbackground, colframe=cellborder]
\prompt{In}{incolor}{26}{\boxspacing}
\begin{Verbatim}[commandchars=\\\{\}]
\PY{n}{np}\PY{o}{.}\PY{n}{linalg}\PY{o}{.}\PY{n}{inv}\PY{p}{(}\PY{n}{A}\PY{p}{)}\PY{n+nd}{@b}
\end{Verbatim}
\end{tcolorbox}

            \begin{tcolorbox}[breakable, size=fbox, boxrule=.5pt, pad at break*=1mm, opacityfill=0]
\prompt{Out}{outcolor}{26}{\boxspacing}
\begin{Verbatim}[commandchars=\\\{\}]
array([[  3.1645799 ],
       [ 10.68135715],
       [ -2.81420873],
       [ 11.71101009],
       [-21.99811914],
       [-10.87303593]])
\end{Verbatim}
\end{tcolorbox}
        
    \hypertarget{solving-the-linear-system-of-equation-ax-b-using-linear_solverab}{%
\section{Solving the linear system of equation Ax = b, using
linear\_solver(A,b)}\label{solving-the-linear-system-of-equation-ax-b-using-linear_solverab}}

    \begin{tcolorbox}[breakable, size=fbox, boxrule=1pt, pad at break*=1mm,colback=cellbackground, colframe=cellborder]
\prompt{In}{incolor}{27}{\boxspacing}
\begin{Verbatim}[commandchars=\\\{\}]
\PY{n}{x} \PY{o}{=} \PY{n}{linear\PYZus{}solver}\PY{p}{(}\PY{n}{A}\PY{p}{,}\PY{n}{b}\PY{p}{)}
\end{Verbatim}
\end{tcolorbox}

    \hypertarget{converting-the-solution-x-to-a-pandas-dataframe}{%
\section{Converting the solution, x to a pandas
dataframe}\label{converting-the-solution-x-to-a-pandas-dataframe}}

    \begin{tcolorbox}[breakable, size=fbox, boxrule=1pt, pad at break*=1mm,colback=cellbackground, colframe=cellborder]
\prompt{In}{incolor}{28}{\boxspacing}
\begin{Verbatim}[commandchars=\\\{\}]
\PY{n}{x\PYZus{}file} \PY{o}{=} \PY{n}{pd}\PY{o}{.}\PY{n}{DataFrame}\PY{p}{(}\PY{n}{x}\PY{p}{)}
\end{Verbatim}
\end{tcolorbox}

    \hypertarget{storing-the-solution-to-a-csv-file-x.csv}{%
\section{Storing the solution to a csv file,
``x.csv''}\label{storing-the-solution-to-a-csv-file-x.csv}}

    \begin{tcolorbox}[breakable, size=fbox, boxrule=1pt, pad at break*=1mm,colback=cellbackground, colframe=cellborder]
\prompt{In}{incolor}{29}{\boxspacing}
\begin{Verbatim}[commandchars=\\\{\}]
\PY{n}{x\PYZus{}file}\PY{o}{.}\PY{n}{to\PYZus{}csv}\PY{p}{(}\PY{l+s+s2}{\PYZdq{}}\PY{l+s+s2}{x.csv}\PY{l+s+s2}{\PYZdq{}}\PY{p}{,}\PY{n}{index}\PY{o}{=}\PY{k+kc}{None}\PY{p}{,}\PY{n}{header}\PY{o}{=}\PY{k+kc}{None}\PY{p}{)}
\end{Verbatim}
\end{tcolorbox}

    \hypertarget{displaying-the-solution-using-linear_solverab}{%
\section{Displaying the solution, using
linear\_solver(A,b)}\label{displaying-the-solution-using-linear_solverab}}

    \begin{tcolorbox}[breakable, size=fbox, boxrule=1pt, pad at break*=1mm,colback=cellbackground, colframe=cellborder]
\prompt{In}{incolor}{30}{\boxspacing}
\begin{Verbatim}[commandchars=\\\{\}]
\PY{n}{x}
\end{Verbatim}
\end{tcolorbox}

            \begin{tcolorbox}[breakable, size=fbox, boxrule=.5pt, pad at break*=1mm, opacityfill=0]
\prompt{Out}{outcolor}{30}{\boxspacing}
\begin{Verbatim}[commandchars=\\\{\}]
array([[  3.1645799 ],
       [ 10.68135715],
       [ -2.81420873],
       [ 11.71101009],
       [-21.99811914],
       [-10.87303593]])
\end{Verbatim}
\end{tcolorbox}
        
    \begin{center}\rule{0.5\linewidth}{0.5pt}\end{center}

    \hypertarget{example}{%
\section{Example:}\label{example}}

\hypertarget{if-a-is-not-positive-definite}{%
\section{If A is NOT positive
definite:}\label{if-a-is-not-positive-definite}}

    \hypertarget{reading-input-file-a_not_positive_definite.csv-for-the-coefficient-matrix-a}{%
\section{Reading input file ``A\_NOT\_Positive\_Definite.csv'' for the
coefficient matrix
A}\label{reading-input-file-a_not_positive_definite.csv-for-the-coefficient-matrix-a}}

\hypertarget{this-can-be-modified-by-the-user}{%
\section{(This can be modified by the
user)}\label{this-can-be-modified-by-the-user}}

    \begin{tcolorbox}[breakable, size=fbox, boxrule=1pt, pad at break*=1mm,colback=cellbackground, colframe=cellborder]
\prompt{In}{incolor}{31}{\boxspacing}
\begin{Verbatim}[commandchars=\\\{\}]
\PY{n}{A} \PY{o}{=} \PY{n}{pd}\PY{o}{.}\PY{n}{read\PYZus{}csv}\PY{p}{(}\PY{l+s+s2}{\PYZdq{}}\PY{l+s+s2}{A\PYZus{}NOT\PYZus{}Positive\PYZus{}Definite.csv}\PY{l+s+s2}{\PYZdq{}}\PY{p}{,}\PY{n}{header}\PY{o}{=}\PY{k+kc}{None}\PY{p}{)}
\end{Verbatim}
\end{tcolorbox}

    \hypertarget{converting-the-a-matrix-from-pandas-dataframe-to-numpy-array}{%
\section{Converting the A matrix from pandas dataframe to numpy
array}\label{converting-the-a-matrix-from-pandas-dataframe-to-numpy-array}}

    \begin{tcolorbox}[breakable, size=fbox, boxrule=1pt, pad at break*=1mm,colback=cellbackground, colframe=cellborder]
\prompt{In}{incolor}{32}{\boxspacing}
\begin{Verbatim}[commandchars=\\\{\}]
\PY{n}{A} \PY{o}{=} \PY{n}{A}\PY{o}{.}\PY{n}{to\PYZus{}numpy}\PY{p}{(}\PY{n}{dtype} \PY{o}{=} \PY{n}{np}\PY{o}{.}\PY{n}{float64}\PY{p}{)}
\end{Verbatim}
\end{tcolorbox}

    \hypertarget{reading-input-file-b_not_positive_definite.csv-for-the-rhs-b}{%
\section{Reading input file ``b\_NOT\_Positive\_Definite.csv'' for the
RHS
b}\label{reading-input-file-b_not_positive_definite.csv-for-the-rhs-b}}

\hypertarget{this-can-be-modified-by-the-user}{%
\section{(This can be modified by the
user)}\label{this-can-be-modified-by-the-user}}

    \begin{tcolorbox}[breakable, size=fbox, boxrule=1pt, pad at break*=1mm,colback=cellbackground, colframe=cellborder]
\prompt{In}{incolor}{33}{\boxspacing}
\begin{Verbatim}[commandchars=\\\{\}]
\PY{n}{b} \PY{o}{=} \PY{n}{pd}\PY{o}{.}\PY{n}{read\PYZus{}csv}\PY{p}{(}\PY{l+s+s2}{\PYZdq{}}\PY{l+s+s2}{b\PYZus{}NOT\PYZus{}Positive\PYZus{}Definite.csv}\PY{l+s+s2}{\PYZdq{}}\PY{p}{,}\PY{n}{header}\PY{o}{=}\PY{k+kc}{None}\PY{p}{)}
\end{Verbatim}
\end{tcolorbox}

    \hypertarget{converting-the-b-vector-from-pandas-dataframe-to-numpy-array}{%
\section{Converting the b vector from pandas dataframe to numpy
array}\label{converting-the-b-vector-from-pandas-dataframe-to-numpy-array}}

    \begin{tcolorbox}[breakable, size=fbox, boxrule=1pt, pad at break*=1mm,colback=cellbackground, colframe=cellborder]
\prompt{In}{incolor}{34}{\boxspacing}
\begin{Verbatim}[commandchars=\\\{\}]
\PY{n}{b} \PY{o}{=} \PY{n}{b}\PY{o}{.}\PY{n}{to\PYZus{}numpy}\PY{p}{(}\PY{n}{dtype} \PY{o}{=} \PY{n}{np}\PY{o}{.}\PY{n}{float64}\PY{p}{)}
\end{Verbatim}
\end{tcolorbox}

    \hypertarget{solving-the-linear-system-of-equation-ax-b-using-linear_solverab}{%
\section{Solving the linear system of equation Ax = b, using
linear\_solver(A,b)}\label{solving-the-linear-system-of-equation-ax-b-using-linear_solverab}}

    \begin{tcolorbox}[breakable, size=fbox, boxrule=1pt, pad at break*=1mm,colback=cellbackground, colframe=cellborder]
\prompt{In}{incolor}{35}{\boxspacing}
\begin{Verbatim}[commandchars=\\\{\}]
\PY{n}{x} \PY{o}{=} \PY{n}{linear\PYZus{}solver}\PY{p}{(}\PY{n}{A}\PY{p}{,}\PY{n}{b}\PY{p}{)}
\end{Verbatim}
\end{tcolorbox}

    \begin{Verbatim}[commandchars=\\\{\}]
Cholesky(A): A matrix is NOT a positive definite matrix
    \end{Verbatim}

    \begin{Verbatim}[commandchars=\\\{\}, frame=single, framerule=2mm, rulecolor=\color{outerrorbackground}]
An exception has occurred, use \%tb to see the full traceback.

\textcolor{ansi-red}{SystemExit}

    \end{Verbatim}

    \begin{center}\rule{0.5\linewidth}{0.5pt}\end{center}

    \hypertarget{example}{%
\section{Example:}\label{example}}

\hypertarget{if-ax-b-is-having-unique-solution}{%
\section{If Ax = b is having unique
solution:}\label{if-ax-b-is-having-unique-solution}}

    \hypertarget{reading-input-file-a_unique_solution.csv-for-the-coefficient-matrix-a}{%
\section{Reading input file ``A\_UNIQUE\_SOLUTION.csv'' for the
coefficient matrix
A}\label{reading-input-file-a_unique_solution.csv-for-the-coefficient-matrix-a}}

\hypertarget{this-can-be-modified-by-the-user}{%
\section{(This can be modified by the
user)}\label{this-can-be-modified-by-the-user}}

    \begin{tcolorbox}[breakable, size=fbox, boxrule=1pt, pad at break*=1mm,colback=cellbackground, colframe=cellborder]
\prompt{In}{incolor}{36}{\boxspacing}
\begin{Verbatim}[commandchars=\\\{\}]
\PY{n}{A} \PY{o}{=} \PY{n}{pd}\PY{o}{.}\PY{n}{read\PYZus{}csv}\PY{p}{(}\PY{l+s+s2}{\PYZdq{}}\PY{l+s+s2}{A\PYZus{}UNIQUE\PYZus{}SOLUTION.csv}\PY{l+s+s2}{\PYZdq{}}\PY{p}{,}\PY{n}{header}\PY{o}{=}\PY{k+kc}{None}\PY{p}{)}
\end{Verbatim}
\end{tcolorbox}

    \hypertarget{converting-the-a-matrix-from-pandas-dataframe-to-numpy-array}{%
\section{Converting the A matrix from pandas dataframe to numpy
array}\label{converting-the-a-matrix-from-pandas-dataframe-to-numpy-array}}

    \begin{tcolorbox}[breakable, size=fbox, boxrule=1pt, pad at break*=1mm,colback=cellbackground, colframe=cellborder]
\prompt{In}{incolor}{37}{\boxspacing}
\begin{Verbatim}[commandchars=\\\{\}]
\PY{n}{A} \PY{o}{=} \PY{n}{A}\PY{o}{.}\PY{n}{to\PYZus{}numpy}\PY{p}{(}\PY{n}{dtype} \PY{o}{=} \PY{n}{np}\PY{o}{.}\PY{n}{float64}\PY{p}{)}
\end{Verbatim}
\end{tcolorbox}

    \hypertarget{reading-input-file-b_unique_solution.csv-for-the-rhs-b}{%
\section{Reading input file ``b\_UNIQUE\_SOLUTION.csv'' for the RHS
b}\label{reading-input-file-b_unique_solution.csv-for-the-rhs-b}}

\hypertarget{this-can-be-modified-by-the-user}{%
\section{(This can be modified by the
user)}\label{this-can-be-modified-by-the-user}}

    \begin{tcolorbox}[breakable, size=fbox, boxrule=1pt, pad at break*=1mm,colback=cellbackground, colframe=cellborder]
\prompt{In}{incolor}{38}{\boxspacing}
\begin{Verbatim}[commandchars=\\\{\}]
\PY{n}{b} \PY{o}{=} \PY{n}{pd}\PY{o}{.}\PY{n}{read\PYZus{}csv}\PY{p}{(}\PY{l+s+s2}{\PYZdq{}}\PY{l+s+s2}{b\PYZus{}UNIQUE\PYZus{}SOLUTION.csv}\PY{l+s+s2}{\PYZdq{}}\PY{p}{,}\PY{n}{header}\PY{o}{=}\PY{k+kc}{None}\PY{p}{)}
\end{Verbatim}
\end{tcolorbox}

    \hypertarget{converting-the-b-vector-from-pandas-dataframe-to-numpy-array}{%
\section{Converting the b vector from pandas dataframe to numpy
array}\label{converting-the-b-vector-from-pandas-dataframe-to-numpy-array}}

    \begin{tcolorbox}[breakable, size=fbox, boxrule=1pt, pad at break*=1mm,colback=cellbackground, colframe=cellborder]
\prompt{In}{incolor}{39}{\boxspacing}
\begin{Verbatim}[commandchars=\\\{\}]
\PY{n}{b} \PY{o}{=} \PY{n}{b}\PY{o}{.}\PY{n}{to\PYZus{}numpy}\PY{p}{(}\PY{n}{dtype} \PY{o}{=} \PY{n}{np}\PY{o}{.}\PY{n}{float64}\PY{p}{)}
\end{Verbatim}
\end{tcolorbox}

    \hypertarget{solving-the-linear-system-of-equation-ax-b-using-linear_solverab}{%
\section{Solving the linear system of equation Ax = b, using
linear\_solver(A,b)}\label{solving-the-linear-system-of-equation-ax-b-using-linear_solverab}}

    \begin{tcolorbox}[breakable, size=fbox, boxrule=1pt, pad at break*=1mm,colback=cellbackground, colframe=cellborder]
\prompt{In}{incolor}{40}{\boxspacing}
\begin{Verbatim}[commandchars=\\\{\}]
\PY{n}{x} \PY{o}{=} \PY{n}{linear\PYZus{}solver}\PY{p}{(}\PY{n}{A}\PY{p}{,}\PY{n}{b}\PY{p}{)}
\end{Verbatim}
\end{tcolorbox}

    \hypertarget{converting-the-solution-x-to-a-pandas-dataframe}{%
\section{Converting the solution, x to a pandas
dataframe}\label{converting-the-solution-x-to-a-pandas-dataframe}}

    \begin{tcolorbox}[breakable, size=fbox, boxrule=1pt, pad at break*=1mm,colback=cellbackground, colframe=cellborder]
\prompt{In}{incolor}{41}{\boxspacing}
\begin{Verbatim}[commandchars=\\\{\}]
\PY{n}{x\PYZus{}file} \PY{o}{=} \PY{n}{pd}\PY{o}{.}\PY{n}{DataFrame}\PY{p}{(}\PY{n}{x}\PY{p}{)}
\end{Verbatim}
\end{tcolorbox}

    \hypertarget{storing-the-solution-to-a-csv-file-x_unique_solution.csv}{%
\section{Storing the solution to a csv file,
``x\_UNIQUE\_SOLUTION.csv''}\label{storing-the-solution-to-a-csv-file-x_unique_solution.csv}}

    \begin{tcolorbox}[breakable, size=fbox, boxrule=1pt, pad at break*=1mm,colback=cellbackground, colframe=cellborder]
\prompt{In}{incolor}{42}{\boxspacing}
\begin{Verbatim}[commandchars=\\\{\}]
\PY{n}{x\PYZus{}file}\PY{o}{.}\PY{n}{to\PYZus{}csv}\PY{p}{(}\PY{l+s+s2}{\PYZdq{}}\PY{l+s+s2}{x\PYZus{}UNIQUE\PYZus{}SOLUTION.csv}\PY{l+s+s2}{\PYZdq{}}\PY{p}{,}\PY{n}{index}\PY{o}{=}\PY{k+kc}{None}\PY{p}{,}\PY{n}{header}\PY{o}{=}\PY{k+kc}{None}\PY{p}{)}
\end{Verbatim}
\end{tcolorbox}

    \hypertarget{displaying-the-solution-using-linear_solverab}{%
\section{Displaying the solution, using
linear\_solver(A,b)}\label{displaying-the-solution-using-linear_solverab}}

    \begin{tcolorbox}[breakable, size=fbox, boxrule=1pt, pad at break*=1mm,colback=cellbackground, colframe=cellborder]
\prompt{In}{incolor}{43}{\boxspacing}
\begin{Verbatim}[commandchars=\\\{\}]
\PY{n}{x}
\end{Verbatim}
\end{tcolorbox}

            \begin{tcolorbox}[breakable, size=fbox, boxrule=.5pt, pad at break*=1mm, opacityfill=0]
\prompt{Out}{outcolor}{43}{\boxspacing}
\begin{Verbatim}[commandchars=\\\{\}]
array([[1.],
       [0.],
       [0.],
       [0.],
       [0.],
       [0.]])
\end{Verbatim}
\end{tcolorbox}
        
    \begin{center}\rule{0.5\linewidth}{0.5pt}\end{center}

    \hypertarget{example}{%
\section{Example:}\label{example}}

\hypertarget{if-a-is-not-a-matrix-square}{%
\section{If A is NOT a matrix
square}\label{if-a-is-not-a-matrix-square}}

    \hypertarget{reading-input-file-a_non_square.csv-for-the-coefficient-matrix-a}{%
\section{Reading input file ``A\_NON\_SQUARE.csv'' for the coefficient
matrix
A}\label{reading-input-file-a_non_square.csv-for-the-coefficient-matrix-a}}

\hypertarget{this-can-be-modified-by-the-user}{%
\section{(This can be modified by the
user)}\label{this-can-be-modified-by-the-user}}

    \begin{tcolorbox}[breakable, size=fbox, boxrule=1pt, pad at break*=1mm,colback=cellbackground, colframe=cellborder]
\prompt{In}{incolor}{44}{\boxspacing}
\begin{Verbatim}[commandchars=\\\{\}]
\PY{n}{A} \PY{o}{=} \PY{n}{pd}\PY{o}{.}\PY{n}{read\PYZus{}csv}\PY{p}{(}\PY{l+s+s2}{\PYZdq{}}\PY{l+s+s2}{A\PYZus{}NON\PYZus{}SQUARE.csv}\PY{l+s+s2}{\PYZdq{}}\PY{p}{,}\PY{n}{header}\PY{o}{=}\PY{k+kc}{None}\PY{p}{)}
\end{Verbatim}
\end{tcolorbox}

    \hypertarget{converting-the-a-matrix-from-pandas-dataframe-to-numpy-array}{%
\section{Converting the A matrix from pandas dataframe to numpy
array}\label{converting-the-a-matrix-from-pandas-dataframe-to-numpy-array}}

    \begin{tcolorbox}[breakable, size=fbox, boxrule=1pt, pad at break*=1mm,colback=cellbackground, colframe=cellborder]
\prompt{In}{incolor}{45}{\boxspacing}
\begin{Verbatim}[commandchars=\\\{\}]
\PY{n}{A} \PY{o}{=} \PY{n}{A}\PY{o}{.}\PY{n}{to\PYZus{}numpy}\PY{p}{(}\PY{n}{dtype} \PY{o}{=} \PY{n}{np}\PY{o}{.}\PY{n}{float64}\PY{p}{)}
\end{Verbatim}
\end{tcolorbox}

    \hypertarget{reading-input-file-b_non_square.csv-for-the-rhs-b}{%
\section{Reading input file ``b\_NON\_SQUARE.csv'' for the RHS
b}\label{reading-input-file-b_non_square.csv-for-the-rhs-b}}

\hypertarget{this-can-be-modified-by-the-user}{%
\section{(This can be modified by the
user)}\label{this-can-be-modified-by-the-user}}

    \begin{tcolorbox}[breakable, size=fbox, boxrule=1pt, pad at break*=1mm,colback=cellbackground, colframe=cellborder]
\prompt{In}{incolor}{46}{\boxspacing}
\begin{Verbatim}[commandchars=\\\{\}]
\PY{n}{b} \PY{o}{=} \PY{n}{pd}\PY{o}{.}\PY{n}{read\PYZus{}csv}\PY{p}{(}\PY{l+s+s2}{\PYZdq{}}\PY{l+s+s2}{b\PYZus{}NON\PYZus{}SQUARE.csv}\PY{l+s+s2}{\PYZdq{}}\PY{p}{,}\PY{n}{header}\PY{o}{=}\PY{k+kc}{None}\PY{p}{)}
\end{Verbatim}
\end{tcolorbox}

    \hypertarget{converting-the-b-vector-from-pandas-dataframe-to-numpy-array}{%
\section{Converting the b vector from pandas dataframe to numpy
array}\label{converting-the-b-vector-from-pandas-dataframe-to-numpy-array}}

    \begin{tcolorbox}[breakable, size=fbox, boxrule=1pt, pad at break*=1mm,colback=cellbackground, colframe=cellborder]
\prompt{In}{incolor}{47}{\boxspacing}
\begin{Verbatim}[commandchars=\\\{\}]
\PY{n}{b} \PY{o}{=} \PY{n}{b}\PY{o}{.}\PY{n}{to\PYZus{}numpy}\PY{p}{(}\PY{n}{dtype} \PY{o}{=} \PY{n}{np}\PY{o}{.}\PY{n}{float64}\PY{p}{)}
\end{Verbatim}
\end{tcolorbox}

    \hypertarget{solving-the-linear-system-of-equation-ax-b-using-linear_solverab}{%
\section{Solving the linear system of equation Ax = b, using
linear\_solver(A,b)}\label{solving-the-linear-system-of-equation-ax-b-using-linear_solverab}}

    \begin{tcolorbox}[breakable, size=fbox, boxrule=1pt, pad at break*=1mm,colback=cellbackground, colframe=cellborder]
\prompt{In}{incolor}{48}{\boxspacing}
\begin{Verbatim}[commandchars=\\\{\}]
\PY{n}{x} \PY{o}{=} \PY{n}{linear\PYZus{}solver}\PY{p}{(}\PY{n}{A}\PY{p}{,}\PY{n}{b}\PY{p}{)}
\end{Verbatim}
\end{tcolorbox}

    \begin{Verbatim}[commandchars=\\\{\}]
Cholesky(A): A matrix is NOT a square matrix
    \end{Verbatim}

    \begin{Verbatim}[commandchars=\\\{\}, frame=single, framerule=2mm, rulecolor=\color{outerrorbackground}]
An exception has occurred, use \%tb to see the full traceback.

\textcolor{ansi-red}{SystemExit}

    \end{Verbatim}

    \begin{center}\rule{0.5\linewidth}{0.5pt}\end{center}

    \hypertarget{example}{%
\section{Example:}\label{example}}

\hypertarget{if-a-and-b-are-of-incompatible-dimensions}{%
\section{If A and b are of incompatible
dimensions}\label{if-a-and-b-are-of-incompatible-dimensions}}

    \hypertarget{reading-input-file-a_incompatible.csv-for-the-coefficient-matrix-a}{%
\section{Reading input file ``A\_INCOMPATIBLE.csv'' for the coefficient
matrix
A}\label{reading-input-file-a_incompatible.csv-for-the-coefficient-matrix-a}}

\hypertarget{this-can-be-modified-by-the-user}{%
\section{(This can be modified by the
user)}\label{this-can-be-modified-by-the-user}}

    \begin{tcolorbox}[breakable, size=fbox, boxrule=1pt, pad at break*=1mm,colback=cellbackground, colframe=cellborder]
\prompt{In}{incolor}{49}{\boxspacing}
\begin{Verbatim}[commandchars=\\\{\}]
\PY{n}{A} \PY{o}{=} \PY{n}{pd}\PY{o}{.}\PY{n}{read\PYZus{}csv}\PY{p}{(}\PY{l+s+s2}{\PYZdq{}}\PY{l+s+s2}{A\PYZus{}INCOMPATIBLE.csv}\PY{l+s+s2}{\PYZdq{}}\PY{p}{,}\PY{n}{header}\PY{o}{=}\PY{k+kc}{None}\PY{p}{)}
\end{Verbatim}
\end{tcolorbox}

    \hypertarget{converting-the-a-matrix-from-pandas-dataframe-to-numpy-array}{%
\section{Converting the A matrix from pandas dataframe to numpy
array}\label{converting-the-a-matrix-from-pandas-dataframe-to-numpy-array}}

    \begin{tcolorbox}[breakable, size=fbox, boxrule=1pt, pad at break*=1mm,colback=cellbackground, colframe=cellborder]
\prompt{In}{incolor}{50}{\boxspacing}
\begin{Verbatim}[commandchars=\\\{\}]
\PY{n}{A} \PY{o}{=} \PY{n}{A}\PY{o}{.}\PY{n}{to\PYZus{}numpy}\PY{p}{(}\PY{n}{dtype} \PY{o}{=} \PY{n}{np}\PY{o}{.}\PY{n}{float64}\PY{p}{)}
\end{Verbatim}
\end{tcolorbox}

    \hypertarget{reading-input-file-b_incompatible.csv-for-the-rhs-b}{%
\section{Reading input file ``b\_INCOMPATIBLE.csv'' for the RHS
b}\label{reading-input-file-b_incompatible.csv-for-the-rhs-b}}

\hypertarget{this-can-be-modified-by-the-user}{%
\section{(This can be modified by the
user)}\label{this-can-be-modified-by-the-user}}

    \begin{tcolorbox}[breakable, size=fbox, boxrule=1pt, pad at break*=1mm,colback=cellbackground, colframe=cellborder]
\prompt{In}{incolor}{51}{\boxspacing}
\begin{Verbatim}[commandchars=\\\{\}]
\PY{n}{b} \PY{o}{=} \PY{n}{pd}\PY{o}{.}\PY{n}{read\PYZus{}csv}\PY{p}{(}\PY{l+s+s2}{\PYZdq{}}\PY{l+s+s2}{b\PYZus{}INCOMPATIBLE.csv}\PY{l+s+s2}{\PYZdq{}}\PY{p}{,}\PY{n}{header}\PY{o}{=}\PY{k+kc}{None}\PY{p}{)}
\end{Verbatim}
\end{tcolorbox}

    \hypertarget{converting-the-b-vector-from-pandas-dataframe-to-numpy-array}{%
\section{Converting the b vector from pandas dataframe to numpy
array}\label{converting-the-b-vector-from-pandas-dataframe-to-numpy-array}}

    \begin{tcolorbox}[breakable, size=fbox, boxrule=1pt, pad at break*=1mm,colback=cellbackground, colframe=cellborder]
\prompt{In}{incolor}{52}{\boxspacing}
\begin{Verbatim}[commandchars=\\\{\}]
\PY{n}{b} \PY{o}{=} \PY{n}{b}\PY{o}{.}\PY{n}{to\PYZus{}numpy}\PY{p}{(}\PY{n}{dtype} \PY{o}{=} \PY{n}{np}\PY{o}{.}\PY{n}{float64}\PY{p}{)}
\end{Verbatim}
\end{tcolorbox}

    \hypertarget{solving-the-linear-system-of-equation-ax-b-using-linear_solverab}{%
\section{Solving the linear system of equation Ax = b, using
linear\_solver(A,b)}\label{solving-the-linear-system-of-equation-ax-b-using-linear_solverab}}

    \begin{tcolorbox}[breakable, size=fbox, boxrule=1pt, pad at break*=1mm,colback=cellbackground, colframe=cellborder]
\prompt{In}{incolor}{53}{\boxspacing}
\begin{Verbatim}[commandchars=\\\{\}]
\PY{n}{x} \PY{o}{=} \PY{n}{linear\PYZus{}solver}\PY{p}{(}\PY{n}{A}\PY{p}{,}\PY{n}{b}\PY{p}{)}
\end{Verbatim}
\end{tcolorbox}

    \begin{Verbatim}[commandchars=\\\{\}]
linear\_solver(A,b): Dimensions of A and b are NOT compatible.
    \end{Verbatim}

    \begin{Verbatim}[commandchars=\\\{\}, frame=single, framerule=2mm, rulecolor=\color{outerrorbackground}]
An exception has occurred, use \%tb to see the full traceback.

\textcolor{ansi-red}{SystemExit}

    \end{Verbatim}

    \begin{center}\rule{0.5\linewidth}{0.5pt}\end{center}


    % Add a bibliography block to the postdoc
    
    
    
\end{document}
